\chapter{Introducing the research participants and the corpus}\label{RuDe}

As laid out in Chapter \ref{CM}, code-mixing/switching is typically observed in everyday peer-to-peer conversations in bilingual communities. Data collection in code-mixing/switching studies thus involve the collecting of recorded samples of bilingual day-to-day speech\footnote{Some scholars, e.g., \citet{torres-travis}, refer to this type of data as “spontaneous”, or “naturally occurring”, bilingual speech.} as well as the building of a corpus of these recordings and their transcripts \citep[cf.][]{adamou-16}. According to \citet[][42]{backus-two-1996}, the vast majority of studies analysing bilingual speech are corpus-based although other data collection procedures are also utilised in the field \citep[see e.g.,][]{kootstra-etal12,bullock-toribio-gullberg-etal}. \citet[][]{backus-two-1996} has noted that all of the then current studies into code-switchin/mixing were ``at least partially based on an actual corpus of spoken language data'' (p. 42). This observation may well be still valid today although a bulk of experimental data have been gathered since then. The studies reported in this book are no exceptions to this tendency and also draw on a corpus of Russian-German bilingual speech, which was collected amongst Russian-German communities in Germany.  

This chapter introduces the Russian-German bilinguals who participated in the data collection and explains why the speech of Russian German youths and young adults became the focus of the current research. The chapter describes Russian Germans as an ethnic group by providing information on their sociolinguistic history and their situation after repatriation to Germany. I will argue that in a situation of repatriation, a purely generational approach to the repatriates' bilingual language use may fail if patterns of their bilingual acquisition are not taken into consideration. I will demonstrate that although the immigration generation may be a reliable predictor of the bilingual ability, it is often overridden by specific paths of language development, particularly prior to immigration. I will therefore describe the respondents as a group and individually. 

I will start by embedding Russian-German repatriation in a context of immigration to Germany from the Soviet Union and its successor states. I will then lay out the sociolinguistic history of Russian Germans in the twentieth century, focusing on the development of their languages and the roles thereof in the community. I will also outline the considerations involved in selecting this group for data collection and particularly its members of the so-called intermediate-generation. The subsequent section will be concerned with the Russian-German youths and young adults who participated in the research as respondents and will then dwell on the respondent selection criteria and the process of respondent recruitment. I will then present the participants as a group and describe each of them in relation to the social networks, in which the recorded conversations took place. These recorded speech samples formed the corpus that was subject to analyses reported in the subsequent chapters of this book. The final section will close this chapter with a conclusion summarising the main aspects of the research participants.

\section{Immigration to Germany from the Soviet Union and its successor states}
The political and economic changes in Eastern Europe and Central Asia in the 1980s and 1990s, including perestroika, the dissolution of the Soviet Union as well as the economic policies of liberalisation in the post-Soviet successor states, led to social instability and economic hardships. These conditions sparked massive migration flows within and from the former Soviet Union. Most, but not all migration flows of the time belong to diaspora migration \citep{heleniak03}. The beginning of emigration in the perestroika period was marked by liberalisation of the Soviet emigration policy in 1987, which had as a consequence that particularly representatives of the German and the Jewish diaspora started leaving the Soviet Union for permanent residence abroad. This process was facilitated by the immigration policies of the countries of their historical origin, which granted them a privileged status for immigration \citep[cf.][]{deTinguy}. Crucially, in addition to the economic instability of perestroika, members of these groups faced political and economic discrimination in the pre-perestroika period. Along with these two groups, representatives of other ethnicities also contributed to permanent emigration from the Soviet Union and the post-Soviet successor states in the wake of socio-economic uncertainties of the 1990s. Most of them settled in the countries of Northern and Western Europe, North America and Oceania. Among the immigration countries, Germany experienced the largest influx of former Soviet citizens; its extent was paralleled, but not equalled by Israel.\footnote{The number of ethnic Germans who immigrated to Germany from the Soviet Union and the post-Soviet successor states between 1988 and 2000 amounts to approximately two million people \citep{lederer97}, whereas Israel received around one million of Jewish immigrants from these countries in the same time period \citep{tolts}.} Overall, three groups of individuals contributed to the immigration from the Soviet Union and the post-Soviet successor states: ethnic Germans, persons of Jewish origin and members of other ethnic groups \citep[cf.][]{brehmer07}.

Ethnic Germans from the Soviet Union and its successor states constitute the largest group of immigrants in Germany. The strong migration flow can be explained by the (West) German policy to regard all ethnic Germans living in Central and Eastern Europe as potential citizens. This policy was based on the (West) German Expellees' and Refugees' Law (\textit{Bundesvertriebenen- und Flüchtlingsgesetz}) of 1953 as well as on an extensive interpretation of the Constitution (\textit{Grundgesetz}) of 1949 (Article 116), which defines as citizens not only former citizens of Germany (within the borders of 1937), but also ethnic German refugees on the West German territories \citep{muenz03}. In 1957, ethnic Germans in the Soviet Union, but also Albania, Bulgaria, China, Czechoslovakia, Hungary, Poland, Romania and Yugoslavia were officially proclaimed German nationals and began to be referred to as \textit{Aussiedler} `repatriates'. Upon being granted, the \textit{Aussiedler} status guaranteed them the same access to benefits as had been conferred on post-war expellees as well as the right to acquire German citizenship immediately after arrival to (West) Germany \citep{muenz03}. Already between 1950 and 1988, 62,023 persons of German descent left the Soviet Union for residence in Germany, but the mass emigration of Germans from the Soviet Union began after the liberalisation of the Soviet emigration policy in 1987. The following decade saw an unprecedented immigration flow of approximately two million \textit{russlanddeutsche Aussiedler} (`German repatriates from Russia') and accompanying family members without German ancestry \citep{lederer97}. After a change in the legislation, repatriates who emigrated to Germany after January 1st, 1993 began to be officially called \textit{Spätaussiedler} `late repatriates'. A more widely used term is \textit{Russlanddeutsche} `Russian Germans' (with \textit{russkie nemcy} being its Russian equivalent, \citealt[cf.][]{meng-protas}). Because this term is a self-designated name, I will use it throughout this thesis.

Although the term “Russian Germans” may imply that members of this group have a full command of the German language, including German dialects spoken in the former Soviet Union, this may not always be the case. As has been widely reported, Russian Germans exhibit varied patterns of bilingual language acquisition and use prior to emigration \citep{berend98,meng01,riehlTA,worbs-etal-13}. In this regard, a differentiation ought to be made between the pre-war generations and the post-war generations: while the pre-war generations were German-dominant at the time of emigration, the post-war generations' primary or only language was in most cases Russian (see section \ref{ru-de} for further details). Notably, the post-war generations made up the lion's share of Germany's Russian Germans in 2011\footnote{2011 was the second year of the data collection in this project.} \citep[41]{worbs-etal-13}. This fact allows to conclude that the majority of Russian Germans can be well considered part of Germany's Russian-speaking community \citep[cf.][]{meng01,roll03,brehmer07}.

Jewish immigrants constitute the second group of Russian-speaking immigrants in Germany. The resolution of the Conference of the Interior Ministers of the Federal States (\textit{Innenministerkonferenz}) from January 9th, 1991, added persons of Jewish descent from the former Soviet Union and their family members to the list of quota refugees. This measure, in connection with the Law Relating to Humanitarian Aid for Refugees (\textit{Gesetz über Maßnahmen für im Rahmen humanitärer Hilfsaktionen aufgenommenen Flüchtlinge, HumHiG}), enabled persons of Jewish descent from the post-Soviet states, excepting the three Baltic states, to immigrate to Germany. Hence, 209,134 Jewish immigrants and their family members from the post-Soviet states came to Germany between 1993 and 2018 \citep[96]{bamf}. The addition of 8,535 persons of Jewish origin who immigrated to Germany before 1993 results in a total influx of 217,669 people \citep[]{bamf}. As Russian is either the first or the primary language of Jewish immigrants in Germany \citep[cf.][166]{brehmer07}, they are considered part of Germany's Russian-speaking community in its full amount (for an overview and further literature on Jewish immigrants in Germany, see \citealt{remennickTA}). 

The last group of Russian-speaking immigrants includes citizens of the post-Soviet successor states who are living in Germany. Whereas the majority of them are foreign nationals, some of them have in the meanwhile acquired German citizenship. These individuals include au pairs, labourers, students, scientists, spouses in mixed marriages, etc. The varied legal and social status of these migrants challenges their systematic quantification \citep[cf.][166]{brehmer07}. Table \ref{tab:3:1} reports the numbers of citizens of the post-Soviet states living in Germany in 2018. As can be seen from the table, Russian citizens make up the largest part of migrants from these countries. As in the case of the aforementioned Jewish community, Russian is the first or the second language for Russian citizens \citep[cf.][]{marten-riessler}. Regarding the citizens of the other states listed in Table \ref{tab:3:1}, we cannot rule out the possibility that citizens of Belarus, Ukraine and Kazakhstan have a fluent command of Russian and citizens of Moldova and the Transcaucasian states have at least some knowledge of Russian \citep{gasimov-paper}.\footnote{The volume edited by \citet{gasimov-book} offers an outline of Russification processes in the Russian empire and the Soviet Union, covering their sociohistorical  aspects.} The reason for this assumption is the Soviet language policy to promote Russian as ``a single language in the formation of a unified, industrialized nation state'' \citep[1]{grenoble-2003}. According to the 1989 Soviet census, 81 per cent of the population reported fluency in Russian as either their first or their second language, although Russians comprised only one half of the population \citep[cf.][2]{grenoble-2003}. We could thus assert that at least for some part of non-Russian immigrants from the post-Soviet successor states, Russian may be a lingua franca for communication between fellow immigrants \citep[cf.][]{levkovych}. Yet, as \citet[167]{brehmer07} correctly notes, it is hardly possible to draw valid conclusions about language use from governmental statistics, such as in Table \ref{tab:3:1}, he therefore considers nationals of the post-Soviet states living in Germany, with the exception of Russian citizens, as potential speakers of Russian.

\begin{table}
\begin{small}
\begin{tabular}{l r}
	\midrule
	    \addlinespace[2mm]
			Country	&Number of individuals \\ \addlinespace[2mm]
			\midrule
			\addlinespace[2mm]
			Russia	&254 325\\
			Ukraine	&141 350\\
			Kazakhstan &46 740\\
			Armenia &27 275\\
			Azerbaijan &26 270\\
			Georgia &25 775\\
			Belarus &22 980\\
            Moldova &20 375\\
			\addlinespace[2mm]
			Total &565 090\\
			\addlinespace[2mm]
			\midrule
\end{tabular}
\end{small}
\caption{\textit{Nationals of the post-Soviet states (without the Baltic states) living in Germany in 2018 \citep[adopted from][276--278]{bamf}.}} \label{tab:3:1}
\end{table}

Overall, in the period between 1952 and 2018, at least 2.8 million Russian speakers migrated from the Soviet Union and its successor states to Germany, of whom only 2.5 per cent arrived in Germany before 1988. These migrants include 2.5 million of \textit{(Spät-)Aussiedler}, 17.5 thousand of Jewish immigrants and 254 thousand of Russian nationals living in Germany. (Together with the citizens of Ukraine, Kazakhstan and Belarus registered in Germany, a total of at least 3 million people were living in Germany in 2018 who were likely to have a fluent command of Russian.\footnote{This count could not consider the former nationals of the Soviet Union and its successor states who have acquired the German citizenship as well as the deceased persons.}) Russian speakers thus constitute one of the largest linguistic minorities in Germany. However, it should be kept in mind that Germany's Russian-speaking community is highly heterogeneous, consisting of persons with diverse social and ethnic backgrounds (although being united by similar cultural experiences in the Soviet Union and its successor states, \citealt[cf.][]{gasimov-paper, levkovych}). On this account, the case studies reported in the subsequent chapters investigate the speech of only one group, i.e., \textit{russlanddeutsche (Spät-)Aussiedler}, being by far the largest group among Germany's Russian speakers.

\section{Russian Germans and their languages prior to emigration}{\label{ru-de}}
Most Russian Germans arrived in Germany through 1990s and the beginning of 2000s. Prior to their emigration, their linguistic situation was similar to that of other linguistic minorities in the Soviet Union and its successor states. \citet[]{berend98} describes it as stable German-Russian bilingualism, involving regular code-switching and code-mixing (p. 3). In a language proficiency survey involving 130 ethnic Germans from the post-Soviet states, she finds that most of them were proficient in Russian and German, usually a German dialect, but on rare occasions standard German. However, as was mentioned in the previous section, their language repertoires varied considerably, depending on the generation to which they belonged. Several studies \citep[e.g.,][]{dietz-hilkes1993, berend98, meng01, riehlTA} have noted that Russian Germans born before World War II (WWII) reported and demonstrated the ability to speak a Russian German dialect and Russian, whereas the post-war generations had no or a very limited command of Russian German, and some knowledge of Standard German, which they learnt at school. \citet{berend98} observes a correlation between her respondents' age and their multilingual competence: the younger the respondents, the higher their proficiency in Russian and Standard German and the lower their knowledge of a German dialect (p. 55, cf. Table \ref{tab:3:2}). The speakers' competence in Standard German also positively correlated with the duration of education (pp. 56--57). It must be noted in this regard that 8-year secondary education had been extended to 10-year education throughout the country by the 1960s, and we can certainly assume that in areas with large proportions of ethnic Germans, they learnt Standard German as a foreign language at school. However, considering the fact that foreign language teaching in the Soviet Union was aimed at teaching only basic comprehension skills, school leavers were barely fluent in standard German \citep[cf.][309]{ivanovaitivyaeva}.

Two main factors contributed to the pre-war generation's ability to speak at least one variety of German: for one thing, most  Russian Germans belonging to these generations grew up in German-speaking cities and settlements \citep[cf.][]{berend98}; for another thing, the Soviet language policy of the 1920s allowed and even prescribed the use of German as the language of administration and a means of instruction in the public schools of the country's German-speaking areas (such as the Volga German Autonomous Soviet Socialist Republic), though sometimes both German and Russian were used in these domains \citep[cf.][]{meng01,riehlTA}. However, the situation changed radically already in the late 1920s and early 1930s \citep[cf.][]{mukhina}: German clergy fell victim to the Soviet policy of elimination of religion, and some wealthy German farmers and well-to-do peasants, who were then referred to as kulaks and considered class enemies, were prosecuted and banished to Kazakhstan or Siberia in consequence of the dekulakisation policy. Further repressions followed: already before WWII, German communities in the European part of the Soviet Union, being accused of collaborating with Nazi Germany, were transported to Siberia and Kazakhstan and forced to live there in special guarded settlements (\textit{Kommandaturaufsicht}); during WWII, many of the deportees, mostly males but also females, were conscripted into labour columns (\textit{Trudarmee} or \textit{Trudarmija}) and sent to labour camps in the north of Russia and the Ural region. Although the police surveillance in the guarded settlements was abandoned in 1955, it is not until 1964 that Soviet Germans were partially rehabilitated and allowed to return to their former homes. Since the deportations of the 1930s, Russian began to oust German from the public sphere. Hostile attitudes toward Germans and all things German, including the language, persisted in the Soviet Union for decades, long after the end of WWII, and resulted in the socio-economic discrimination against the German minority. Although the situation eased slightly during the 1960s, the German language, being still stigmatised, was relegated to the private domain (\citealt[49]{berend98}; \citealt[21]{blankenhorn}). 

Another factor which facilitated the shift in language dominance in the after-war generations is urbanisation. \citet{berend-riehl} report that 85 per cent of Russian Germans were living in rural areas in 1926, and the proportion sank to 50 per cent in 1979 (p. 23). This statistic is corroborated by \citeauthor{meng01}'s (\citeyear[83]{meng01}) study of Russian Germans' linguistic biographies: of 42 interviewed Russian Germans born between 1948 and 1972, only one person reported growing up in a German settlement, 19 interviewees grew up in cities, and 22 respondents spent their childhood in multilingual settlements. Hence, the aforementioned historical events, particularly the dissolution of the established German-speaking settlements, and socio-economic factors directly affected Soviet Germans' language use: after WWII most of them were living outside German settlements and thus ``in an unstable linguistic situation, which [in many cases] led to a gradual attrition of German and to its eventual loss'' (\citealt[20]{berend98}, my translation).

The correlation between the generation to which a Russian-German person belongs and a specific language acquisition pattern holds for all the recently reported surveys of Russian Germans' language use, even despite some heterogeneity across the studies with regard to two aspects. Some researchers \citep[e.g.,][]{berend98,meng01} collected data in Germany, whereas other scholars \citep[e.g.,][]{blankenhorn,riehlTA} gathered data from respondents in Russia. Moreover, studies vary slightly as to the classification of generations, or age groups, particularly with regard to the parameter “birth year”. For example, \citet{meng01} identifies six age groups and specifically differentiates between two groups, or generations, of children: ``children of preschool age'' (\textit{Vorschulkinder}), born between 1984 and 1992, and ``school children'' (\textit{Schulkinder}), born between 1976 and 1986. In contrast, \citet{riehlTA} distinguishes between four generations, the youngest of which includes Russian Germans born after 1975. In Table \ref{tab:3:2}, I reproduce \citeauthor{riehlTA}'s (\citeyear[]{riehlTA}) approach to relating the four generations of the interviewed Russian Germans to reported language acquisition patterns. 

\begin{table}
\begin{small}
\begin{tabular}{l p{7.2em}<{\raggedright}p{7.2em}<{\raggedright}p{7.2em}<{\raggedright}p{7.2em}<{\raggedright}}
	\midrule
	    \addlinespace[2mm]
			& 1. Generation born before 1932 & 2. Generation born between 1932 and 1952 & 3. Generation born between 1952 and 1975 & 4. Generation born after 1975 \\ \addlinespace[2mm] 	\midrule
			\addlinespace[2mm]
		L1 & Russian German dialect	& Russian German dialect with markers of attrition; Russian	& Russian; (passive competence of German dialect) 	& Russian\\
		\addlinespace[2mm]
		L1 & Standard German	& Standard German (rudimentary)	& 	& \\
		\addlinespace[2mm]
		L2 & Russian (partly as an interlanguage)	& 	& Standard German as an inter- language 	& Standard German as an inter- language at various levels\\
		\addlinespace[2mm]
		\midrule
\end{tabular}
\end{small}
\caption{\textit{Language competence across generations in the Russian German diaspora in Russia (adopted from \citealt{riehlTA}).}}\label{tab:3:2}
\end{table}

The gradual shift from German to Russian, which is characteristic of the Russian German post-war generations, is easily identifiable in the table. Interestingly, although the generation born between 1932, namely before WWII, and 1952 is distinguished by simultaneous bilingualism, they, in their turn, ``did not actively transmit German to their children'' \citep[22]{riehlTA}. The next generation, i.e., generation three, roughly corresponds to \citeauthor{meng01}'s (\citeyear{meng01}) generation four, called ``young parents, born between 1948 and 1972, mainly between 1955 and 1969'' (``Junge Eltern, geboren zwischen 1948 und 1972, meist zwischen 1955 und 1969'', p. 20). Albeit half of this generation reports German (usually a dialect) to be their first language, and one third claims to have simultaneously acquired German and Russian in their early childhood, 93 per cent of Russian Germans belonging to this generation asserted a better command of Russian than German prior to emigration \citep[36]{meng01}. It therefore comes as no surprise that they spoke almost exclusively Russian with their children \citep[35]{meng01}. Their children -- they correspond to \citeauthor{meng01}'s (\citeyear{meng01}) group two (“schoolchildren”) and \citeauthor{riehlTA}'s (\citeyear{riehlTA}) generation four, born after 1975 -- as is shown by these studies, almost invariably acquired only Russian as their first language, albeit they may have been exposed to either a German dialect or Standard German in their childhood \citep[cf.][35]{meng01}. Indeed, four of my research participants who repatriated at the age of seven or older have reported remembering their speaking German (probably a dialect) in their families. It is impossible to say though what their proficiency in that language, or its dialect, was and how the input they received could be described in terms of its quality and quantity. Additionally, more than a half of my research participants attended secondary school in the Soviet Union or in any of the post-Soviet successor states prior to their emigration to Germany. Therefore, they might have learnt German as a foreign language and may thus have had (limited) exposure to standard German. However, considering the duration of the secondary education that they received in their countries of origin and the aforementioned fact that the focus in foreign language teaching up to the mid 1990s had been largely on passive skills, we may hardly expect that they were fluent in standard German. The members of this group were either children or adolescents when they resettled in Germany with their parents. In the next section, my focus will be on these speakers as a specific immigrant group.

\section{Research participants: Russian-German youths and young adults}{\label{junge-ru-de}}
\subsection{1.5-generation Russian-German immigrants}
As outlined in the previous section, young Russian Germans, born in the late 1970s, 1980s and early 1990s, were socialised in Russian \citep[cf.][106]{meng01} and integrated into the Russian-speaking community \citep[cf.][275]{roll03}. As most young Russian Germans were fully accepted in the social contexts prior to emigration, in the majority they had ``neither any knowledge of German language and culture nor of modern or even postmodern German society'' (\citealt[272]{roll03}; for further details, see \citealt{dietz-roll198}). Learning the German language was thus the main challenge that they faced upon emigration and a key factor influencing their socioeconomic status and prospects. This challenge was even greater for children, who had not yet developed a full command of Russian, their first language \citep[cf.][106-152]{meng01}. This demanding situation is typical for immigrants of the intermediate, or 1.5, generation.  These are individuals who moved to a host country as children or adolescents (between the ages of 5 to 18), usually following their parents and/or other family members. (\citealt{backus1999,backus06} uses the term “intermediate generation” to refer to these individuals, cf. chapter (\ref{social-factors}), whereas \citealt{remennickTA}, employs the shortcut “1.5”.)  According to \citet{remennickTA}, no agreement exists among scholars on the age bracket defining the 1.5-generation, but most researchers acknowledge the unique character of immigrant experience typical of this generation: it differs from the experience of both the first generation (their parents) and the second generation (children born in the host country). 

1.5-generation immigrants are well suited to collecting bilingual speech samples because most of them are proficient in both languages (cf. \citealt[43]{backus-patterns-1992}; \citealt[36-38]{halmari-government-1997}; \citealt[160-167]{boumans-syntax-1998}). For example, Turkish 1.5-generation immigrants in The Netherlands have been found to have neither a preference for Turkish, the origin country language, nor a preference for Dutch, the host country language \citep[201]{backus06}. This view is very much in line with research on nativelike attainment, according to which nativelikeness, defined as L2 learners' performance that corresponds to the range observed with native subjects, is found even among learners whose age at immigration is in the late teens \citep[121]{birdsong2009}. Another argument in favour of collecting speech of 1.5-generation immigrants for investigating code-mixing is a relatively high frequency of code-mixing in their speech. For example, in a 1999 paper \citeauthor{backus1999} reports that of the three generations of Turkish immigrants in The Netherlands, representatives of the 1.5-generation produce twice as many mixed sentences as the second-generation immigrants and about three times as many as the first-generation immigrants (p. 263).\footnote{\citet{goldbach05} shows that the speech of the first-generation immigrants (Russian and Ukrainian nationals without German ancestry living in Berlin) may also involve intensive code-switching and code-mixing.} 

These observations motivated the data collection for building a bilingual Russian-German corpus in the first place. The goal was to record samples of naturally occurring bilingual speech produced by Russian Germans of the intermediate generation. The first phase in the process of gathering these data was the recruitment and selection of  eligible individuals.

\subsection{Recruiting and selecting research participants}

The selection of participants was guided by four parameters: ({i}) immigrant generation, ({ii}) use of Russian on daily basis, ({iii}) social integration in Germany, and ({iv}) duration of residence. While the two former parameters introduced indirect means for control of proficiency in Russian, the latter two parameters sought to provide indirect means for control of their knowledge of German. 

The key parameter for selecting suitable bilingual speakers was the generation of immigration: Russian Germans belonging to the 1.5 generation of immigrants were identified as eligible for participating in this research. This consideration rested on the foregoing observation that 1.5-generation immigrants are usually fluent in the origin country language. However, as origin country languages may be affected by attrition \citep[on the development of Russian in Russian-German immigrants of the 1.5 generation, see][]{meng01,meng-protas}, I introduced a further requirement, namely, regular use of Russian in daily life. 

The third criterion pertained to the immigrants' social integration. The underlying assumption here was that participation in the host country's major social institutions, such as the labour force and educational institutions, enhances host country language proficiency. For example, in a study of the social integration of Russian Jewish immigrants in Israel, \citet{remennick03} found that Hebrew competence correlated with the immigrant's occupation type: ``respondents with skilled or professional jobs reported good/very good Hebrew three times more often than did respondents with unskilled jobs and \textit{five times more often than did unemployed or retired respondents}'' (p.32, my emphasis). Therefore, only if Russian German youths and young adults were attending or had graduated from German general secondary or vocational education and training institutions, they qualified as eligible participants in this research. Virtually all 1.5-generation Russian Germans automatically satisfy this criterion as the lion's share of them received general secondary education and another portion was involved in vocational education. Finally, to additionally ensure the respondents' proficiency in German, a further condition was introduced as a selection criterion: respondents were expected to have stayed in Germany for a period of approximately ten years, or longer. In a nutshell, in order to meet the selection criteria of this research, an ideal prospective participant was supposed to be an intermediate-generation Russian-German immigrant who had been staying in Germany for a period of at least ten years, was receiving or had received secondary, and possibly vocational, or higher, education there and used Russian in their daily communication at home or in their social network. 

In order to ensure the greatest possible degree of authenticity and naturalness in the recorded interactions, the respondents were selected as members of either existing social networks or naturally occurring social groups. 
This criterion was given primacy over the language biographical parameters introduced above when applied to informal multi-party conversations. Whenever the majority of bilinguals engaging in such a conversation satisfied the aforementioned language biographical criteria, the recording was included in the corpus.

The corpus falls into two parts, each corresponding to a different kind of setting in which the audio-recorded interactions took place. One type of interaction subsumed informal conversations recorded by recruited young adults in their social networks, whereas the other type covered peer group interactions among youths in a school setting. I will first dwell on recruiting the respondents who recorded naturally occurring conversations with their friends and family members and will then elaborate on selecting younger respondents, whose conversations were recorded in school.

Each individual from the first group received a written invitation to participate in the study. The invitation contained general information about the project and instructions for audio-recording of conversations. It specifically asked the respondents to record themselves and their communication partners in spontaneous everyday interaction, so as to emphasise the importance of recording conversations that reflect natural language use. Moreover, the invitation recommended the respondent to engage in conversation with their peers, i.e., friends and/or relatives of a similar age and with the same Russian-German background. It also instructed the participant to inform beforehand the conversation partners about the audio-recording and to obtain their consent on it. Finally, it guaranteed all the persons involved that the audio-recordings will be used only for research purposes and in an anonymous format.

A first attempt to recruit Russian German young adults consisted in sending the invitation to the Youth and Student Association of Germans from Russia (\textit{Jugend- und Studentenring der Deutschen aus Russland e. V.}), which is the youth organisation of the Homeland Association of Germans from Russia (\textit{Landsmannschaft von Deutschen aus Russland e.V.}), and the Youth Migration Service of the Caritas Association of Freiburg (\textit{Jugendmigrationsdienst des Caritasverbandes Freiburg-Stadt e.V.})\footnote{The mentioned organisations are all incorporated associations (\textit{eingetragene Vereine, e.V.}).}. Whilst my request addressed to the former organisation remained without response, the latter organisation kindly provided me with a list of persons willing to participate in the study. Two respondents, Svetlana and Elena (here and below I use fictitious names), could be recruited in this way. Another participant, Tatyana, was recruited from my personal Hanover network. Notwithstanding our previous acquaintance, she remained unaware of my specific interest in gathering speech samples with code-mixing until the end of the data collection.

My efforts to recruit young Russian Germans in Lahr, a city in Baden with a large proportion of Russian Germans \citep[cf.][]{roll03}, were futile, despite the mediation of the local group of the Homeland Association of Germans from Russia. The contacted persons either felt reluctant to participate in the research study or, being second generation immigrants, demonstrated only a limited command of Russian and thus did not meet the study's selection criteria. However, Lahr's local group played a major role in establishing contact with Marina, an respondent from Villingen-Schwenningen. Finally, I distributed posters publicising the project on the University of Freiburg campus. Two students with the Russian-German background could be recruited thereby: Irina and Olga. They were both living in the aforementioned city of Lahr and were strongly rooted in the local Russian-German networks. With their help, overall six Russian-German young adults volunteered for data collection. Irina and Olga were involved in at least one audio-recorded conversation, which they had held with members of their social networks. A speech sample of another Russian-German bilingual, Julia, was taken from the Regensburg corpus of Slavic-German bilinguals (\textit{Das Regensburger Korpus slavisch-deutscher Bilingualer}); in this corpus, collected by \citet{rebislav}, each speech sample is supplemented by relevant sociolinguistic information about the speaker.

As regards the other group, i.e., Russian-German youths, they were all senior students of \textit{Die Kaufmännischen Schulen \textnormal{/} das Integrierte Berufliche Gymnasium Lahr} (`the commercial schools / vocational secondary school of Lahr'). At this level of education, completion of a second foreign-language programme is compulsory. In order to pass this course successfully or with little effort, some Russian German students, although they may have acquired Russian as an L1, chose Russian as their second foreign language subject. Yet, other Russian German students elected the Russian course for further reasons. Some sought, for example, to acquire literacy in Russian and to thus enhance their language proficiency. The identified motivations should however not be conceived of as mutually exclusive. Although many Russian German students had acquired Russian as an L1, their proficiencies in Russian differed. This variability is presumably determined by a shift, at least in some of the individuals, to German as a primary language.

The director of the school and the Russian teacher allowed me to establish contact with the students and engage them in an elicitation task. The task was to be carried out in freely formed discussion groups of three or four students. Free choice of partners allowed the students to consider the natural bonds and affiliations existing between them while forming groups. This measure was intended to facilitate lively and natural group discussions. After the groups were formed, the students were confronted with the task, which consisted in comparing and contrasting aspects of Russian and German cultures. The group discussions, stimulated by a set of prepared questions, were recorded for subsequent analysis. The questions contained mixed utterances such as \textit{Kak mo\v{z}no ocharakterisovat' Russlanddeutsche doma im Vergleich zu den einheimischen Deutschen?} `How can Russian Germans be characterised at home when compared to indigenous Germans?' (the first three words of the question as well as the word \textit{doma} `at home' are Russian), or \textit{Silvester i Weihnachten, was wisst Ihr von den Traditionen in Russland und Deutschland? Kak prasdnujut Russlanddeutsche simnie prasdniki?} `New Year's Eve and Christmas, what do you know about the traditions in Russia and Germany? How do Russian Germans celebrate the winter holidays?' (the first sentence is in German except for the Russian coordinator \textit{i} `and', the second sentence is in Russian except for the autonymic ethnonym Russian Germans). 

The analysis revealed that six of the thirteen students had high fluency in Russian and frequently switched the languages during the group discussions. All of them had learnt Russian in early childhood, although the extent of their literacy was often limited to the input received from the Russian class at school. With the teacher's permission, I made several visits to the school, during which I engaged with these selected students in group conversations by asking them questions but leaving room for extensive peer interaction. Crucially, my speech, like theirs, was distinguished by extensive code-switching and code-mixing. In the course of the conversations, the students frequently left the given subjects and started to talk about other things. Hence, these interactions are comparable with conversations recorded by young adults in their social networks.

Upon the recordings, respondents were asked to complete a paper self-report questionnaire, which was designed in order to gather basic sociolinguistic information. The requested information concerned the participants' migration and language acquisition histories, their language preference as well as language use in their social networks. Since the questionnaire data reported below are based on self-report, as any other self-report based data, they should be regarded with caution. Although self-report measures assume that people answer the questions honestly, respondents may sometimes be inclined to give incorrect responses, wanting to make a good impression \citep[cf.][208]{social-psychology}. For example, already in her \citeyear{poplack-sometimes-1980} pioneering paper, Poplack showed that bilingual speakers occasionally over- and underrated their language proficiency, even though the great majority of the respondents provided self-report estimates which were compatible with the objectively observed proficiency rates. However, because the focus of the present research is on explaining patterns of Russian-German code-mixing by relating them to facts of usage, cognitive processing and structural regularities, rather than social factors as such, the reported survey results are intended as a demonstration of the respondents' sociolinguistic backgrounds, which serves as a basis for comparing the investigated group with other bilingual communities, as to their preferences for specific patterns of bilingual speech, in general, and code-mixing, in particular.

The remainder of this chapter entails a presentation of the questionnaire results. It first characterises the study participants as a group and then introduces them as members of their social networks.

\section{Informants as a group}
This section presents and compares the respondents in terms of their age, migration history and linguistic memories about their language development. All the respondents' names are fictitious. It is furthermore necessary to note that all the respondents, with one exception, are female. This situation was not intended and owes to the strenuousness of the recruitment process.

\subsection{Age and migration history}
As detailed in Table \ref{tab:3:3}, the respondents were youths and young adults, with their age ranging between 18 and 35 years (mean 25 years). Their age at immigration varied from 4 to 21 years. Those individuals who were 7 to 18 years of age at the time of immigration qualify as immigrants of the intermediate generation. These respondents constitute the majority of the bilinguals sampled in the corpus. Among the few exceptions to this tendency are three first-generation and two second-generation immigrants. The former group comprises three participants who moved to Germany at the age of 19 to 21 years. The decision to include these speakers in the corpus rested on two considerations: First, the period of their residence in Germany had exceeded ten years by the time of recording. Second, all of them were fluent in German: while Valentina and Marina reported growing up bilingual, Inna developed her proficiency in German after the immigration and that enabled her to enrol at the University of Freiburg. The latter group includes two second-generation immigrants, Larisa and Alex. Larisa was born in Germany into a Russian German family, whereas Alex arrived in Germany being a four-year-old. Their speech samples were included in the corpus because, being rooted in the Russian-German community of Lahr, they still maintained Russian as a community language and regularly used it in their networks. Regarding residence duration, one respondent, Rita, did not satisfy the study's residence requirement, according to which each respondent was supposed to have stayed in Germany for approximately 10 years. The reasons for including her in the corpus are discussed in conjunction with her sociolinguistic profile in section \ref{subgroups}.


%\begin{tabular}{ p{5em}<{\raggedleft}p5em}<{\centering}p{5em}<{\\centering}p{5em}<{\\centering}p{5em}<{\\centering}}


\begin{table} 
\begin{small}
		\begin{tabular}{p{6.5em}<{\raggedright}p{5.5em}<{\raggedright}p{5.5em}<{\raggedright}p{5.5em}<{\raggedright}p{6.5em}<{\raggedright}} \midrule
			\addlinespace[2mm]
			Informant's fictitious name & Age at recording & Age at immigration & Duration of residence & Place of living\\ \addlinespace[2mm] \midrule
			\addlinespace[2mm]
		Larisa & 21 & local-born & 21 & Lahr\\
		Alex & 20 & 4 & 16 & Lahr\\
		Olga & 24 & 7 & 17 & Lahr\\
		Alina & 24 & 8 & 16 & Hanover\\
		Vika & 20 & 8 & 12 & Lahr\\
		Vera & 19 & 8 & 11 & Lahr\\	
		Svetlana & 27 & 10 & 17 & Freiburg\\
		Nataša & 24 & 10 & 12 & Freiburg\\
		Elena & 28 & 11 & 17 & Freiburg\\
		Ira & 27 & 11 & 16 & Freiburg\\
		Nadja & 19 & 11 & 8 & Lahr\\
		Julia & 21 & 12 & 9 & Regensburg\\
		Tanja & 30 & 14 & 16 & Hanover\\
		Rita & 18 & 15 & 3 & Lahr\\
		Irina & 27 & 17 & 10 & Lahr\\
		Olesja & 30 & 18 & 12 & Lahr\\
		Inna & 29 & 19 & 10 & Lahr\\
		Valentina & 31 & 20 & 11 & Lahr\\
		Marina & 35 & 21	& 14 & Villingen-Schwenningen\\
		\addlinespace[2mm]
		\midrule
	\end{tabular}
\end{small}
	\caption{\textit{Informants by age at immigration.}}\label{tab:3:3}
\end{table}

As to the geographical distribution of the bilinguals sampled in the corpus, it follows from the table that more than half of them were living in Lahr, a city with a high proportion of Russian German repatriates. The other speakers represented in the corpus were residents of other German cities.

\subsection{Language acquisition history}
The study participants report learning Russian as their first language or one of their first languages. As to German, the situations and ages of its acquisition varied across the participants. These aspects of the acquisition of German are given in Table \ref{tab:3:4}. The participants were asked whether they remembered speaking German by the age of seven years. Of 19 participants, five individuals reported having an oral proficiency in German at that age, most probably in a Russian-German dialect. They learnt German through the natural interaction with their parents (referred to in the table as 
“family”) and can thus be considered early bilinguals. The others had no memories of speaking German at the age of seven years, although five respondents (marked by asterisks in the table) remembered hearing single German words, sayings or songs. One respondent reported learning German before immigration through interactions with her relatives and friends (“environment” in the table).

\begin{table} 
\begin{small}
		\begin{tabular}{p{7em}<{\raggedright}p{7em}<{\raggedright}p{7.5em}<{\raggedright}p{7em}<{\raggedright}} \midrule
			\addlinespace[2mm]
		
			Informant's fictitious name & Age at immigration & Acquisition prior to immigration & Acquisition upon immigration \\ \addlinespace[2mm] \midrule
			\addlinespace[2mm]
		Larisa & local born & family & \\
		Alex & 4 & family & family\\
		Olga & 7 & & environment\\
		Alina & 8 & * & environment\\
		Vika & 8 & & GFL\\
		Vera & 8 & & GFL\\	
		Svetlana & 9 & & environment\\
		Nataša & 10 & family & family\\
		Elena & 11 & environment, GFL & environment\\
		Ira & 11 & GFL & GFL \\
		Nadja & 11 & GFL & GFL \\
		Julia & 12 & GFL & ?\\
		Tanja & 14 & GFL* & environment\\
		Rita & 15 & GFL & GFL\\
		Irina & 17 & GFL* & GFL\\
		Olesja & 18 & GFL* & GFL\\
		Inna & 19 & GFL* & environment\\
		Valentina & 20 & family & \\
		Marina & 21 & family & \\
		\addlinespace[2mm]
		\midrule
	\end{tabular}
\end{small}
	\vspace{3mm}
	\caption{\textit{Informants' acquisition of German by situation types prior and upon immigration. (The typical situations in which the respondents learnt German include the family [in early childhood], the natural environment and classes of German as a foreign language (GFL) at school in their country of origin, or in a language school in Germany. The asterisk marks respondents with memories of fragmentary exposure to German in early childhood. The ReBiSlav corpus did not provide information on Julia's linguistic development.)}}\label{tab:3:4}
\end{table}

The respondents who were living in their home countries at the age of eleven or older were learning German as a foreign language in the classroom (“GFL” in the table). After migration to Germany, the participants either acquired German through natural interactions with speakers in their environment or through visiting specific language courses (“GFL” in the table). A comparison of Tables \ref{tab:3:3} and \ref{tab:3:4} makes two tendencies visible: First, individuals who moved to Germany either at a very early age or at least 16 years before the study took place, i.e., in the early 1990s, manifest the tendency to acquiring German through natural interactions. Second, individuals who immigrated to Germany at the age of seven years or older and less than 16 years before the study took place, i.e., in the late 1990s, tend to have learnt German in classroom context.

In a nutshell, the participants of the study manifest the following acquisition patterns for German: five respondents learnt German in the family and named it one of their first languages, three respondents learned German in an informal context beyond the family, and ten respondents acquired German in a primarily formal context. However, as we can see in Table \ref{tab:3:4}, the identified patterns of acquisition sometimes overlap and the exposure to German in the early childhood may also play a role in the later acquisition of the language. I therefore present detailed sociolinguistic backgrounds of the individual respondents below.

\section{Informant subgroups}\label{subgroups}

This section draws a portrait of each of the 19 respondents by making use of the sociolinguistic information collected by the aforementioned questionnaire.\\

\noindent \textit{Irina and Olga}\\
Irina and Olga responded to the announcement poster distributed at the university campus Freiburg. The young women knew each other prior to the recordings. At the time of data gathering they were enrolled in the Slavic Department of the University of Freiburg. This fact may be interpreted as a sign of their affinity towards the Russian language and culture. Some of their audio-recordings contained conversations between each other, whereas the others included conversations with members of their individual social networks.

Both Irina and Olga were born into mixed families: their fathers are Russian Germans, their mothers are Russian. As would be expected in this case, their parents spoke Russian to them prior to emigration. Although Irina reported being exposed to fragmentary German input in form of chunks, i.e., individual lexical items, phrases, and songs, she remembers speaking only Russian before the age of seven. Olga, in her turn, asserted that before the age of seven she spoke only Russian. Although Irina and Olga's early language biographies parallel each other to a large extent, their migration histories vary. Olga left Russia in 1995 at the age of seven years, whereas Irina moved to Germany in 2002 at the age of 17 years. Hence, upon arrival Olga learnt German from friends, relatives and at school, i.e., in her natural environment, whereas Irina had to take a German course before attending public school.
Irina considered herself more proficient in Russian than German, whereas Olga's estimation was exactly the opposite. However, in daily-life situations it was easier for both of them to interact in German. Both Irina and Olga were living in Lahr. Being rooted in the large local Russian-speaking community, each of them audio-recorded several hours of conversation within their social networks. 

Irina's Russian is close to standard, and her German is fluent. She has high language awareness and actually tries to avoid language mixing, obviously due to the need to speak standard Russian at the university. Although mixing and switching do occur in her speech, especially when her conversation partners mix languages, she sometimes pauses in the middle of a sentence and looks for a translation equivalent. Olga's language development differs from Irina's. Her German is native-like, but her Russian is also fluent. Her bilingual speech is characterised by frequent code-switching and code-mixing, even when she speaks to Irina.

The bulk of the recordings comprises one-on-one conversations between the two young women, in which they talk about their university and work life as well as mutual acquaintances and friends. Some of their conversations cover such topics as the Russian identity, the histories of Russia and Germany as well as current developments in the Russian politics. One of their interactions involves one of their fellow students. In this conversation, the primary topic is university-related matters and the speakers code-mix a lot. Apart from these conversations, Irina and Olga audio-recorded several interactions in their individual networks. Irina's recordings include a group conversation with Olesja and Valentina, both of whom were also Lahr residents.\\

\noindent \textit{Olesja and Valentina}\\
Olesja and Valentina arrived in Germany in 1999 and 2002 at the ages of 18 and 20 years respectively. Their parents are Russian Germans and they grew up in similar socio-cultural environments. Although Olesja's home town Omsk exceeded Valentina's home town Astana in population size at the time when the women were living there, both cities had a fairly equal proportion of Russian Germans. Yet, despite this fact, Olesja's parents decided to speak only Russian to her, whereas Valentina received an equal amount of input in both German and Russian from her parents and reports growing up bilingual. However, Olesja remembered being exposed to German, even though this input was fragmentary. Upon her arrival in Germany, unlike Valentina, she had to attend a German course.

From the recorded conversation, it can be concluded that Russian is the dominant language for each of the interlocutors. Throughout the conversation Russian is more frequent than German, although both languages are widely used. Olesja's and Valentina's Russian may be described as colloquial with a few non-standard sprinklings. As to German, Valentina appears to be more secure about it than Olesja. She also uses more German in the conversation than Olesja and Irina taken together. Nevertheless, Olesja's productive proficiency in German is adequate, which may be due to the fact that she had lived in Germany for twelve years before the recording took place and had been working as a sales person. The three conversation participants  --  Irina, Olesja and Valentina  --  regularly switch to German when attending to their children, who speak German most of the time.\\ 

\noindent \textit{Olga and Inna}\\
Olga recorded several conversations with her close friend Inna, who was also a Lahr resident. Like Olga, Inna was born into a mixed family: her mother is a Russian German, her father is Russian. Unlike Olga, who arrived in Germany in 1995 at the age of seven years (see above), Inna immigrated as a young adult of 19 years in 2002. Therefore, they demonstrate opposing trends in language proficiency: Olga regards German as her strongest language, whereas Inna considers  Russian to be her better language. Nonetheless, before the age of seven Olga and Inna report Russian to be their primary language for speaking, despite they had received a minor portion of German input.

By the time of recording, Inna had lived in Germany for ten years and identified herself as a Russian. In her daily life, the use of each language is often compartmentalised: Russian is used in the affective domain, i.e., with the family members, relatives, partner and Russian-speaking fellow students, and German is used at the university, at work, but occasionally also with Russian-speaking friends. 

In the recorded conversations they discuss their university life, plans for the future and exchange opinions about mutual acquaintances and friends. Hence, the conversations abound in topic-related code-switching as well as code-mixing.\\

\noindent \textit{Tanja and Alina}\\
I had known Tanja before conducting this study, and after I asked her to audio-record some conversations with her sister Alina, she immediately agreed. Tanja and Alina were born in Uzbekistan to Russian-German parents in 1980 and 1986, respectively. The family language was Russian, but the children also heard some German (presumably a German dialect), as used by their grandparents and other relatives. When the family left Uzbekistan in 1994, Tanja was 14 years of age and her sister was eight years old. The different ages at immigration influenced (and most likely still influence) the sisters' language use and ethnic identities. The elder sister identified herself as a Russian German, whereas the younger sister felt as a German. In her network, Tanja used both languages equally, unlike Alina, for whom the use of Russian was restricted to her parents and grandparents. With her sister, she reported speaking both languages, switching them back and forth. As for their proficiency assessment, Tanja considered herself to be equally fluent in both languages, whereas her sister rated her oral proficiency in German much higher than in Russian. However, the recorded conversations reveal extensive code-switching and code-mixing in each of the sisters' speech. They talk about household chores, flat renovation, their children and mutual acquaintances.\\

\noindent \textit{Marina}\\
Marina is the only respondent from Villingen-Schwenningen, a city at the historical border between the regions of  Baden and Swabia. Marina recorded an interaction at the nursery school where she was working as a teacher and an informal conversation with her Russian-speaking friends. As both of her friends did not fulfil the study's requirement of residence duration, their contributions to the conversation were disregarded in the analysis. Prior to her emigration in 1999 at the age of 21 years, Marina lived in Kazakhstan, where she was born into a Russian German family. A Russian German dialect, standard German and Russian were spoken in her family. She thus grew up bilingual, acquiring Russian and standard German. At the time of the recording she considered herself a Russian German and evaluated her command of Russian as native-like and her proficiency in German as fluent. Based on the recorded conversations, her Russian may be described as a variety very close to the standard. She preferred speaking both languages without separating them as isolated codes, or, putting it in Grosjean's terms, using them in a bilingual mode \citep[cf.][]{grosjean85}. Interestingly, she used Russian with her Russian husband and spoke German with her children. In her social network, both languages were constantly in use. In the recorded conversation, she and her friends talk about their lives in Germany, their children and food.\\

\noindent \textit{Elena, Ira and Nataša}\\
Elena's contact details was provided by the Youth Migration Service of the Caritas Association of Freiburg. Elena recorded several casual interactions with her friends Ira and Nataša. Elena's and Nataša's linguistic memories are similar in several ways: First, they were born into mixed families, with one of the parents being a Russian German and the other parent being a Russian. Second, they remembered only speaking Russian before they started school (although Nataša's mother spoke German to her daughter). Third, they were almost of the same age when they moved to Germany: Nataša was ten years old (year of immigration: 1997) and Elena was eleven years old (year of immigration: 1996). However, at the time of recording, more than fifteen years after their immigration, they manifested diverging tendencies in language use and bilingual ability: Elena reported a regular use of the two languages in her network and equal fluency in both of them, while Nataša tended to speak German slightly more frequently than Elena and regarded her command of German higher than her proficiency in Russian. As to Ira, her parents, though being Russian Germans, used only Russian as a family language. The family arrived in Germany in 1995, when she was eleven years old. Upon her immigration, Ira learnt German predominantly in a language course. According to her report, the extent to which she used Russian in her daily-life interactions prevailed the extent of German used in the same contexts. Although Ira had lived in Germany for twenty years  by the time of data collection, she still preferred Russian to German, but evaluated her proficiency levels in the two languages as equally high. In their conversation, Elena, Ira and Nataša talk about family matters and real estate prices in Freiburg. During their conversation, the friends have their children at the table receiving their meal and repeatedly attend to them.\\

\noindent \textit{Svetlana}\\
As in the above case, I received Svetlana's contact information from the Youth Migration Service of the Caritas Association of Freiburg. At the time of the recording she was a single working mother with a preschooler and was about to marry her fiance, also a Russian German. Being pressed for time, or for any other reasons, she refused to record a casual conversation for me, but allowed me to converse with her. During our conversation, Svetlana (and I) were switching the languages back and forth; that was a natural and preferred way for her to speak Russian. Svetlana's family moved to Germany in 1994, when she was 10 years of age. During her childhood in Russia, her parents spoke to her Russian and German, and she claimed speaking ability in these languages by the age of seven years. Svetlana may thus count as a Russian-dominant bilingual. It is yet unclear whether the variety of German that she remembers speaking was closer to Standard German, or to one of the German dialects spoken in Altai Krai, Russia. At the time of the recording, Svetlana considered herself a German and evaluated her command of German as higher than her command of Russian. Although her Russian was strong, she had experienced several situations in her professional life in which she was unable to use it for professional purposes. This may have led her to think of her Russian competence as poor. In daily communication, Svetlana preferred speaking Russian and German in a bilingual mode. She reports using both languages for communication in most contexts.\\

As stated above, all younger participants of the study, except Julia, were senior students from Lahr. Julia's speech sample was taken from the ReBiSlav corpus \citep[cf.][]{rebislav}. Although the respondents from Lahr knew each other from the Russian class, each of them was linked with one of the two pre-existing groups of friends: Nadja, Rita, Vera and Vika, on the one hand, and Alex and Larisa, on the other hand. These bilingual speakers are described as members of the corresponding social network.\\

\noindent \textit{Nadja, Rita, Vera and Vika}\\
The girls caught my attention already during the elicitation task by being conspicuously loud among their classmates while discussing the task questions. They turned out to be a tightly-knit, enclosed group who stuck together in and after school. The girls regularly met each other individually, or gathered as a group. They celebrated holidays together, and on summer holidays, some of them travelled together. In their clique, they  cultivated an atmosphere of trust and emotional immediacy. As to the clique's language of interaction, Russian was unequivocally the primary language, but code-mixing was the norm. The topics of the recorded conversations range from school matters to their childhood experiences in the countries of the former Soviet Union and Germany to the Fukushima nuclear disaster. 

Nadja, Vera and Vika were all born into mixed families, with one of the parents being a Russian German, and the other parent belonging to another ethnic group. In other words, they have the same ethnic background through one of their parents. By contrast, Rita's ties to Russian Germans are only through friends. Hence, she is the only respondent to contribute to the corpus who has no Russian-German ethnic background. Although Nadja, Vera and Vika spent their childhood in various places, even different countries (Nadja and Vera was born in Kazakhstan, and Vika in Russia), their language biographies and language learning memories are alike. All of them report learning Russian as their first language and not being exposed to German in their childhood. Yet, the three respondents also manifest differences. For example, their ages of arrival in Germany vary: Vera's and Vika's families moved to Germany in 2001 and 2000 respectively, when the girls were eight years old, and Nadja's family arrived in Germany in 2004, when she was eleven. This means that Nadja completed her primary education and the other two received some part of it in the countries of their origin in Russian. Upon immigration to Germany, they all attended German classes for repatriates. At the time of recording, the speakers rated their proficiencies in the two languages as follows: Nadja and Vika reported equally high proficiencies in both languages; Vera considered her German to be better than her Russian, which she rated as good. All of them stated that they preferred speaking the two languages in a bilingual mode, i.e., without separating them.

Rita was born into a Russian family, and arrived in Lahr following her mother. Rita moved to Germany in 2009 at the age of 15 years, but her first encounter with German took place some time earlier: she began learning it before leaving Russia. After English, German was the third language she learnt. By the time of recording, she had lived in Germany only for three years, but her German was more than adequate, particularly in casual interactions. Rita's extremely fast integration into the local community contributed substantially to her fluency in German. Being a very communicative person by nature, not only she had started attending school from her first days in Lahr, but she had also became a core member of the described bilingual clique as well as a member of a sports club, and had a German-speaking partner for some time. In her after-school job, she was involved in communication with German-speaking customers and colleagues. At the time of data collection, her German was slightly dialect-coloured, partly because she had received much input from her stepfather and colleagues, who spoke the local Alemannic dialect to her. Rita's leading position in the group and her exceptional language ability were the primary reasons for including the recordings of her speech in the corpus.\\

\noindent \textit{Alex and Larisa}\\
Alex and Larisa were classmates having a friendly relation without being close friends. They had similar language biographies but entirely different identities and attitudes towards Russian. Both of them were born into mixed families, Alex in Kazakhstan and Larisa in Germany. Both Russian and German were spoken in their families, so that the children grew up bilingual. Larisa is the only study participant who was born in Germany. Alex's family moved to Germany in 1996, when he was four. Therefore, they both count as second-generation immigrants. His parents saw no value in Russian, and German became the primary language at home. At the time time of recording, he considered himself as a German rather than a Russian German and his use of Russian was limited to the communication with his grandparents and a few friends. Unlike Alex, Larisa valued the Russian language and culture. She felt strongly attached to her Russian-German friends and partner and considered herself a Russian German. Larisa reported receiving much of her Russian input from her bilingual clique. Nevertheless, Larisa's and Alex's language proficiencies paralleled each other at the time of recording: they were evidently German-dominant and ranked their German skills higher than their Russian skills.\\

\noindent \textit{Julia}\\
Julia is the only intermediate-generation immigrant from the ReBiSlav corpus \citep{rebislav}. Of all the speakers sampled in the corpus, only her data could be used in this work. Julia's family moved to Germany in 1996, when she was twelve years of age. Russian was her first language, and she reports learning German in Germany. However, at the time of recording she was a German-dominant bilingual, with a limited Russian proficiency. Nevertheless, she reported regularly using Russian in several contexts.\\

As can be seen from the linguistic portraits of the  Russian-German bilinguals sampled in my corpus, their linguistic competences manifest similarities rather than differences, even though they represent immigrants of three generations: first-generation immigrants (two participants), 1.5-generation immigrants (14 participants) and  second-generation immigrants (three participants). Crucially, all of the respondents demonstrated bilingual ability, being fluent in the host-country language and maintaining Russian in their day-to-day interactions. The intergenerational differences in my respondents' language competences and preferences appeared to be minor. This situation is attributed to two circumstances: the existence of tightly knit social networks and early bilingualism. Social networks and communal ties were particularly important for the two second-generation immigrants included in the corpus. Maintaining tight connections to Lahr's Russian-speaking community allowed them to be regularly exposed to Russian and to learn to use it in everyday situations. At the same time, two of the sampled first-generation immigrants reported acquiring both languages at an early age prior to their immigration to Germany. Such a case is generally rare in the context of immigration, but not uncommon in the context of repatriation, provided that the minority language was maintained. That is, the “generational” approach alone cannot account for the patterns of bilingual language use in a situation of repatriation. Therefore, an adequate description of Russian Germans' languages should consider both the generation of immigration and the individual paths of language acquisition.

\section{Data}

The data were collected in two types of setting. The lion's share of the gathered material included informal interactions audio-recorded by research participants in their social networks. These interactions occurred in the participants' everyday situations such as at home, in the street and in a restaurant as well as on train and car rides. Overall, the participants recorded 15.5 hours of naturally occurring speech. While most of the conversations of this type involve two participants, some of them are multi-party conversations; for example: 

\ea
\label{ex:3:1}
(HO100712)\\
 \gll A: ty slyšala čto mara ma- mara maša s danikom \textit{heiraten}?\\
	{} you heard that Mara {} Mara, Maša with Danik {get\textunderscore{}marry}\\ 
\glt \hfill \\ 
 \gll T: gde? kogda?\\
	{} where when\\
	\glt \hfill \\
\gll A: v ijule zags, a v avguste \textit{hochzeit} \\
	{}  in July {registry\textunderscore{}office} and in August wedding\\
\glt \hfill \\

\gll T: tak \textit{kurzfristig}?\\
	{}  so short-term\\
\glt \hfill \\

\gll A: ja sama tol'ko nedavno uslyšala. ona mne desjat' raz rasskazyvala \phantom{m} što oni \textit{heiraten} \textit{dann} \textit{doch} \textit{nicht} \textit{heiraten}\\
	{}  I myself just recently heard she me ten times told {} that they {get{\textunderscore}married} then actually not {get{\textunderscore}married}\\
\glt \hfill \\

\gll T: èto oni rešili vot. mama, kak tam u maši s danikom? \phantom{m} \phantom{m} kakogo čisla u nix zags? \textit{standesamt}?\\
	{} \textsc{ptcl} they decided \textsc{ptcl} mother how there at Maša with Danik {} {} what date with them {registry\textunderscore{}office} {registry\textunderscore{}office}\\
\glt \hfill \\

\gll M: čisla pervogo\\
    {} date first\\
\glt \hfill \\
    
\gll T: ((laughing)) \textit{standesamt} kogda u nix?\\
    {} {} {registry\textunderscore{}office} when with them\\
\glt \hfill \\

\gll M: aah ponjatija ne imeju. vy že mne govorili.\\
    {} {} idea not {I\textunderscore{}have} you \textsc{ptcl} me told\\
\glt \hfill \\

\gll A: net.\\
    {} no\\
\glt \hfill \\

A: `Have you heard that Maša and Danik are getting married?'\\
T: `Where? When?'\\
A: `In the registry office in July, and their wedding is in August.'\\
T: `At such a short notice.'\\
A: `I've heard it myself only recently. She told me ten times that they are \phantom{m} getting married and that they aren't.'\\
T: `Well, they've decided now, you see. Mother, how is it with Maša and \phantom{m} Danik? On what date do they have (the appointment in) the registry office?'\\
M: `Around the first.'\\
T: `The registry office, when do they have it?'\\
M: `Ah, I have no idea. It is one of you who told me.'\\
T: `No.'
\z

\noindent{}This conversation begins as a two-party interaction and develops into a three-party talk. At first, Alina announces her sister Tanya that a familiar couple is getting married, and then Tanya consults their mother about the date of the marriage registration. The language of the conversation, although being basically Russian, is permeated by German single words and multi-word sequences. The speech is thus described as language mixing. At the same time, the passage contains an instance of code-switching: After using the Russian term for the registry office, \textit{zags}, Tanya provides its German equivalent, \textit{Standesamt}, in order for her mother to better understand what she is speaking about. This kind of reiteration is a typical example of code-switching. Crucially, the base language of each turn is Russian, and no change in the conversation's base language takes place between the turns. Even Alina's multiword switch \textit{heiraten, dann doch nicht heiraten} ‘are getting married and then they are not' at the end of her turn does not incite a shift in the language of the conversation. I therefore conceive of the language practice in this passage, just as overall in the corpus, as code-mixing, rather than code-switching.

The other type of data includes recordings of interactions made by the writer. With two exceptions, the conversations in which I was one of the participants or an observer took place in the school setting. A total of ten hours of speech was recorded in this type of participant constellation. All but one interactions recorded by myself were multi-party conversations. The following snatch of informal talk, involving Rita (Ri), Vera (Ve), Vika (Vi) and the writer (Re), illustrates this type of data.

\ea
\label{ex:3:2}
(LS110316)\\
 \gll Ri: èto ja nazyvaetsja ja xotela na ètju- èti \textit{ferien} zarabotat' i \phantom{m} \phantom{nn} vsë \textit{spar}-ovat'. i na \textit{führerschein} i čë? \phantom{mmmmmm} \phantom{nn} \textit{führerschein}? na jogu ščas pojdu [ha-ha \textit{ha-und-em}] \textit{bestellen} \phantom{m} \phantom{nn} ha-ha-ha\\
	{} this I is\textunderscore{}called I wanted for {} this holiday to{\textunderscore}earn and {} {} everything to{\textunderscore}save and for driving\textunderscore{}licence and what {} {} driving\textunderscore{}licence to yoga now will\textunderscore{}go {} H{\&}M order {} {} {}\\
\glt \hfill \\

\gll Vi: [ha-ha \textit{ha-und-em}]\\
	{} {} H{\&}M\\
	\glt \hfill \\
	
\gll Ve: [ha-ha \textit{ha-und-em}]\\
	{} {} H{\&}M\\
\glt \hfill \\

\gll Ri: \textit{alles} \textit{ähnliche} a \textit{führerschein} podoždët\\
	{}  all similar and driving\textunderscore{}licence will\textunderscore{}wait\\
\glt \hfill \\

\gll Re: \textit{aber} \textit{jemand} \textit{hat} \textit{gesagt} \textit{dass} \textit{man} \textit{gar} \textit{nicht} \textit{bei} \textit{ha-und-em} \phantom{mm} \phantom{mm} \phantom{mn} \textit{einkauft}.\\
	{} but someone has said that they \textsc{ptcl} not at H{\&}M {} {} {} shop\\
\glt \hfill \\

\gll Ri: ja sebe èto toka vsjakie \textit{görtel} ili čën'-t' takoe pokupaju
ili \phantom{nn} \textit{balerinas} vsë takoe kak by; čën'-t' [drugoe ne pokupaju]\\
	{} I myself \textsc{ptcl} only various belts or something similar buy or {} {} all that like \textsc{ptcl} something different not buy \\
\glt \hfill \\

\gll Ve: [tam] tam takie prostye vešči možno kupit', a esli vot \phantom{mmm} \phantom{nn}  \textit{während} tam odevat'sja èto\\
    {} there there such simple things possible to\textunderscore{}buy but if \textsc{ptcl} {} {} while there to\textunderscore{}dress this\\
\glt \hfill \\
    
\gll Ri: [da] ne-ne\\
    {} {yes} no\\
\glt \hfill \\

\gll Vi: \textit{ich} \textit{hab} \textit{mir} \textit{ein} \textit{kleid} \textit{für} \textit{fünf} \textit{euro} \textit{bestellt}\\
    {} I have me a dress for five euros ordered\\
\glt \hfill \\

\gll Ri: he-he\\
    {} {}\\
%\glt \hfill \\

\gll Vi: \textit{aber} \textit{nicht} \textit{so} \textit{einfach}; \textit{ist} \textit{einfach} \textit{mit} \textit{rüschchen} \textit{so\textunderscore{}n} \textit{bisschen}\\
    {} but not so simple is simply with frills a bit\\
\glt \hfill \\

\gll Ri: ne esli ja sebe tam čën'-t' pokupaju to vot ètot vot top \phantom{nn} za četyre \textit{euro} ((laughing))\\
    {} no if I myself there something buy then \textsc{ptcl} this \textsc{ptcl} {} {} for four euros\\
\glt \hfill \\

Ri: `So much for my wish to earn (some money) during these holidays \phantom{mn} \phantom{mn} and save everything. For my driving licence. And now? The driving \phantom{mn} licence? Now I am going to take yoga classes, to order H{\&}M articles. \phantom{mn} Ha, ha, ha!'\\
Vi: `Ha, ha, H{\&}M.'\\
Ve: `Ha, ha, H{\&}M.'\\
Ri: `All those things, and the driving licence will wait.'\\
Re: `But somebody said that they never buy anything at H{\&}M.'\\
Ri: `I buy myself only stuff like belts or similar things, or balerinas, stuff \phantom{mn} like that. I don't buy other things.'\\
Ve: `You can buy simple things there, but, while you cannot buy all your \phantom{mn} clothes there.'\\
Ri: `Yeah, no-no.'\\
Vi: `I've ordered myself a dress for five euros.'\\
Ri: `He, he.'\\
Vi: `But it's not that simple; it is simply with some frills.'\\
Ri: `No. If I buy myself anything there, then a top like this for four euros.'
\z

\noindent{}At the beginning of this excerpt from a conversation recorded in a school setting, Rita describes her plans for the upcoming holidays and mentions her incapability to save  money for the driving licence because she spends it on yoga classes and H{\&}M articles. After I expressed my surprise at the fact that despite an earlier statement, the girls buy things at that store, each of the girls downplayed this fact by naming the few articles that they consider purchasable there. While I formulated my surprise in German, Rita and Vera responded to it in Russian, and only Vika phrased her turn in German. In spite of that, Rita continues in Russian, thus sticking to the language of the conversation. Crucially, each of the turns framed in Russian contains German(-origin) words.\footnote{Without additional evidence, it is impossible to decide whether the German lexeme \textit{Euro} is a switch, or a borrowing.} In this regard, the language consultants' speech recorded by myself is similar to the samples recorded by the recruited community members in natural settings.

The recorded conversations in both types of setting contain purely monolingual intervals. Most of the time the base language of such monolingual passages is Russian, but in some situations it is German. For example, in situations involving a German monolingual or a German dominant speaker, the interlocutors switch to German as the base language of interaction. Being extremely rare, these situations barely influence the distribution of languages in the corpus.

Table \ref{tab:3:5} lists the situations in which the data were collected and details the constellations of the conversation participants. Additionally, it specifies the duration of each of the recordings. All the recordings marked with the asterisk correspond to the situations in which the writer was one of the participants. Altogether, these recordings include ten hours of recorded conversation. The duration of the samples recorded by the language consultants in their networks amounts to 15 and a half hours. The bulk of these data (6 and a half hours) includes the interactions between Irina and Olga, university students of Russian. Crucially, including such a large portion of interactions between specific speakers to the corpus did not affect the distribution of the reported patterns of bilingual speech because the conversations between Irina and Olga contained a large number of long Russian monolingual intervals. As these passages were unsuitable for the investigation of language mixing, the scope of this book, the studies reported below draw on the data extracted only from bilingual turns.

\begin{table} 
\begin{small}
		\begin{tabular}{p{6.5em}<{\raggedright}p{5.5em}<{\raggedright}p{5.5em}<{\raggedright}p{5.5em}<{\raggedleft}p{6.5em}<{\raggedright}} \midrule
			\addlinespace[2mm]
			Recording	& Speaker	& Speech situation & Duration of recording & Other people present\\ \addlinespace[2mm] \midrule
			\addlinespace[2mm]
	FI110801* & Svetlana	& home & 1 h 30 min & child\\
    FM120811 &Elena, Ira, Nataša & restaurant, lunch table & 25 min &children\\
    HO100712 & Alina, Tanya	& home & 05 min & \\
    HO100712:01	& Alina, Tanya	& home & 35 min & mother, brother-in-law, children\\
    HO100712:02	& Alina, Tanya & restaurant, lunch table & 30 min  & friend\\
    LA120503:01	& Irina, Olga &restaurant, lunch table & 14 min & \\
    LA120503:02-05 &Irina, Olga &on a train & 2 h 00 min & \\
    LA120503:06	&Inna, Olga	&in a car &48 min &\\
    LA120503:09, LA120503:13 &Inna, Olga &home &3 h 08 min\\
    LA120503:11, LA120503:12 &Inna, Olga &in the street, later Inna's home &1 h 28 min &\\
    LS101221* &Nadya, Rita, Vera, Vika & classroom &37 min &fellow students, teacher\\
    LS110125* &Nadya, Rita &classroom &1 h 21 min &\\
    LS110316* &Rita, Vera, Vika &schoolyard &58 min &\\
    LS110405* &Vera, Alex &classroom &1 h 02 min &\\
    LS110510* &Larisa, Alex &classroom &55 min &\\
    LS110526* &Nadya, Vera &classroom &1 h 22 min &\\
    LS110712* &Larisa &classroom &1 h 12 min &fellow student\\
    LS110714* &Nadya, Rita, Vera &classroom &35 min &\\
    LV120224:01-10, LV120224:13-21 &Irina, Olga &on a train &4 h 02 min &\\
    LV120224:11	&Irina, Olesya, Valentina &home &40 min &their children\\
    LV120224:12	&Irina, Olga &on a train &35 min &fellow student\\
    VS120425* &Marina &restaurant, dinner table &50 min &friends\\	\addlinespace[2mm]
		\midrule
	\end{tabular}
\end{small}
	\caption{\textit{Speech situations of the recordings.}}\label{tab:3:5}
\end{table}

The next step in the construction of the data sets for specific case studies involved the transcription of the recorded speech. As mentioned above, I disregarded the monolingual intervals in the conversations and transcribed only the bilingual passages. I finally extracted Russian sentences containing German lexical items and German sentences containing Russian lexical items. The great majority of the identified items included singly occurring nouns and verbs as well as adjective-modified noun phrases and prepositional phrases. These contexts, with the exception of verbs, yielded the grammatical contexts scrutinised in the remainder of this book. 

\section{Conclusion}
In this chapter, I have described the research participants, who were German repatriates from the former Soviet Union and its successor states. They are traditionally referred to as Russian Germans and constitute  Germany's largest group of Russian speakers. Although German, encompassing the standard language and various Russian German dialects, was the traditional community language of Russia's Germans, they have been undergoing language shift to Russian since the Second World War. Prior to the large-scale repatriation to Germany in the late 1980s and the 1990s, transmission of the minority language in the families had largely ceased. Therefore, the focus of the current research has been on the generations born between the late 1970s and the early 1990, which were labelled here as youths and young adults. These speakers demonstrate high-level proficiencies in both Russian and German, and their bilingual speech is likely to exhibit considerable variability in code-mixing. 

Crucially, unlike the typical intermediate-generation immigrants, who learn the host language after moving to the host country, Russian Germans of the intermediate generation had often been exposed to German prior to their repatriation. A third of the research participants reported having grown bilingual, with German being one of their family, or community, languages prior to the repatriation. Other five respondents claimed that they had received some linguistic input in German in form of chunks, including sayings, songs, and the like. For this reason, the selection of the research participants was guided by their linguistic biographies, although it largely followed the traditional generational approach to immigrant languages, which relates the first, the second, and the intermediate generation to a specific pattern of bilingual language use and dominance. I have argued that careful treatment of the bilingual speaker's linguistic development in the community languages prior and upon their repatriation may ensure a high degree of homogeneity among the gathered speech samples in terms of the amount and quality of language mixing therein as well as more general patterns of bilingual language use.

Finally, I described the methods of data collection. While half of the material used for the corpus construction were informal peer interactions recorded by language consultants in natural speech situations, another part of the data included conversations recorded by the writer, in which he was one of the participants. Solely the bilingual turns of the recorded interactions were subject to transcription and further analysis. The inspection of the other-language elements in sentences framed by the base language of the interaction demonstrated that nouns and their combinations with adjectives and prepositions are among the most frequent items occurring in bilingual sentences, aside from verbs. Therefore, the remainder of this book focuses on the distributional patterns of the other-language nouns in the presented bilingual corpus.