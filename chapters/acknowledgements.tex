\chapter*{Acknowledgements}

This is a slightly revised version of my doctoral dissertation that I defended in 2016 at the University of Freiburg. A lot of people have contributed to the success of this work and I am happy to have the opportunity to thank them. First and foremost, I owe my sincere gratitude to my supervisors Peter Auer and Juliane Besters-Dilger, who, from our first meeting on, were very enthusiastic about my project and supported me from my first days in Freiburg. Your acumen, diligence and insight have been a source of inspiration to me. I thank them and John Nerbonne, the reviewer of my dissertation, for their valuable comments on the previous version of this work. 

 I am also greatly indebted to the \textit{Deutsche Forschungsgemeinschaft, DFG} and the research training group \textit{Graduiertenkolleg} 1624 ``\textit{Frequenzeffekte in der Sprache}'' (Frequency effects in language) for funding this work and to Stefan Pfänder, the former speaker of the graduate school, for invaluable personal support. I would also like to thank my colleagues from the graduate school as well as from the German and the Slavic Department of the University of Freiburg.

I remain deeply indebted to the participants of my research for their invaluable contributions and patience as well as to those who so generously assisted me in establishing contacts with eligible participants, including Olga Held of The \textit{Landsmannschaft von Deutschen aus Russland e.V.} (Homeland Association of Germans from Russia), Group Lahr, and Tabea Maire of The \textit{Jugendmigrationsdienst des Caritasverbandes Freiburg-Stadt e.V.} (Youth Migration Service of the Caritas Association in Freiburg) and particularly to the Russian teachers Bettina Lipinski and Friederike Posega from the \textit{Kaufmännische Schule Integriertes Berufliches Gymnasium} (Occupational and Business High School) of Lahr. Without their help this project would never have been realised.

Many thanks go to Ad Backus, Raffaela Baechler, Pia Bergmann, Javier Caro Reina, Eugenio Goria and Marjoleine Sloos, who were willing to discuss my research, and to James Walker for pointing out an important article to me. I am also grateful to the audiences at the 9-th and 10-th International Symposia on Bilingualism and the SKY conference on language contact in Helsinki for commenting on my work, especially in its early stages. For the assistance and advice on statistical and computational matters I thank Uli Held, Christoph Wolk and Benedikt Szmrecsanyi.

Thanks are also due to Isabelle Léglise and Stefano Manfredi, the editors of the Contact and Multilingualism  series of the Language Science Press, for having accepted the manuscript to this series. I am likewise grateful to one anonymous reviewer of the original manuscript for insightful comments and helpful suggestions.

I would finally like to thank all my friends and family for their support and patience, and particularly Tobi, who not only shared his time with me during all these years but contributed to the success of this project by his keen advice, encouragement and technical help.