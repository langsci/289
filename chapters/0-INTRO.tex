\chapter*{Introduction} %nicht mit einer nummer im verzeichnis
\chaptermark{Introduction} %was oben auf dem header steht
\addcontentsline{toc}{chapter}{Introduction}  %wie es im inhaltsverzeichnis zu nennen

Bilingual speakers often describe language mixing, which is notorious for its variability, as “messy”, but we know at the latest since Shana Poplack's seminal \citeyear{poplack-sometimes-1980} work that variation in this inherently manifold phenomenon is largely orderly. Further, as pointed out first by \citet{bokamba89} and summed up later by \citet[][487]{muysken-etal96}, variation in code-mixing results from an intricate interplay of structural, psycholinguistic and sociolinguistic factors. Looking back on more than three decades of rigorous research into code-mixing, we can state with certainty that despite the intense research activity in the field, many questions about the structure of this variation and the driving forces behind it have remained unanswered. The main question still open concerns the nature of the motivations underlying code-mixing patterns in bilingual speech. This overarching question can be broken down to more specific questions, such as ``What are the explanations for the code-mixing patterns we observe?'', ``How psychologically plausible are they?'' and ``How can various motivations be adequately examined in a testable multilevel fashion?''. In search of answers to these questions, I will investigate code-mixing between two inflecting-fusional languages, Russian and German.

My research draws on a corpus of Russian-German bilingual speech that I recorded in some of Germany's communities of repatriates from the former Soviet Union and its successor states. The speakers sampled in the corpus include, for the most part, immigrants of the intermediate generation. Code-mixing in their speech is mainly of the insertional type. This means that stems or longer constituents from German are regularly inserted in otherwise Russian sentences. Crucially, German multiword and multimorphemic constituents systematically alternate with mixed constituents, consisting of German stems and Russian inflectional suffixes. The bulk of the insertions appearing in the corpus is constituted by German nouns and their combinations with German and Russian adjectives and prepositions. This tendency accords with the observation reported for other bilingual communities that nouns and nominal constituents are among the most frequent insertions of in the discourse framed by the bilinguals' other language. The high rate of code-mixing observed in  contexts involving nouns determined the choice of the specific linguistic phenomena for the distributional and structural analysis. The grammatical contexts in which patterns of mixing were scrutinised thus include the adjective-modified noun phrase, the prepositional phrase and the marking of plural on noun insertions. 

The purpose of this research is to describe  variable code-mixing patterns and account for them in terms of competition among several factors. These include usage frequency, linguistic and discursive context as well as distinctive and overlapping properties of Russian and German. Although most of the foregoing motivations have been discussed, or at least adumbrated, in the literature (e.g., \citealt{myers-scotton-duelling-1993,myers-scotton-contact-2002,backus-two-1996,backus-units-2003,boumans-syntax-1998,muysken-bilingual-2000,}), they have neither been subjected to systematic analysis, nor have they been studied in interaction with each other. In exploring the relationship between these factors and the competing patterns, my approach builds on usage-based approaches to language. These theories integrate the gradience and variability of linguistic structures and a psychologically plausible theory of mental representations and thus provide an adequate framework to examine the structure of code-mixing and the motivations behind it.

This book is organised into six chapters. Chapter \ref{CM} will set the scene for the empirical chapters that follow. It will define the scope of the term “code-mixing”, introduce Muysken's \citeyear{muysken-bilingual-2000} code-mixing typology, outline social factors influencing the patterning of bilingual speech and survey, albeit briefly, the key approaches to insertional code-mixing, including Myers-Scotton's Matrix Language Frame model (\citeyear{myers-scotton-contact-2002,myers-scotton-duelling-1993}) and Backus's unit hypothesis (\citeyear{backus-units-2003}). By doing so, it will emphasise empirical shortcomings in both approaches and suggest possible solution paths. The remainder of the chapter will discuss the code-mixing versus borrowing controversy. Although from a synchronic view, code-mixing and borrowing can theoretically be viewed as a continuum, I will argue that they are different phenomena by virtue of their different distributions in bilingual speech.

Chapter \ref{UBL} will provide the theoretical backdrop for the conducted analyses. I will begin by summarising usage-based exemplar models of language, a central tenet of which is that linguistic structure is represented in the mind as memories of specific language experiences as well as in form of generalisations over these memories. Special emphasis will be given to the role of recurrent multiword sequences and multimorphemic words in language acquisition and language processing because they stand at the centre of my analysis of code-mixing. The chapter will close with a presentation of a usage-based perspective on language variation.


Chapter \ref{RuDe} will introduce the participants of my study, Russian German youths and young adults of an intermediate-immigrant generation. I will demonstrate that in regard to their bilingual abilities and linguistic backgrounds, they constituted a sufficiently homogeneous group so that their speech was well-suited to study insertional code-mixing. The chapter will open with a description of German repatriates from the Soviet Union and its successor states as part of Germany's sizeable Russian-speaking community, and will proceed with an overview of Russian Germans' sociolinguistic history prior to emigration. After outlining the selection criteria for participation and giving details of the participant recruitment, the chapter will introduce the participants first as a group and then individually. Finally, the chapter describes the methods underlying the construction of the corpus and presents the speech situations in which the  conversations were recorded.

Chapters \ref{NP}, \ref{PP} and \ref{PL} will constitute the core of this book, they will present three case studies tapping into variation in code-mixing patterns in specific morphosyntactic contexts. Chapters \ref{NP} and \ref{PP} will investigate code-mixing at the level of syntax, and Chapter \ref{PL} will investigate a phenomenon pertaining to the morphological structure of bilingual speech. Specifically, Chapters \ref{NP} and \ref{PP} will analyse code-mixing in the prepositional phrase and the adjective-modified noun phrase, respectively. In these syntactic contexts, German constituents inserted in otherwise Russian sentences, sometimes referred to as  embedded-language islands, alternate with mixed constituents. The alternating patterns studied in Chapter \ref{PL} will concern German noun insertions which either retain their German plural marking and thus form the so-called internal embedded-language islands, or receive Russian inflectional suffixes and form mixed plurals. 

Each of the three chapters capitalizes on the  distributional properties of a specific structural template. A noun occurring in a given structural template combines with the other part of the template -- a plural-marker, a preposition, or an  adjective -- with varying probabilities, depending on the number of forms participating in the distribution. While the use of nouns in plural contexts is linked to the competition between two forms: the German noun stem,  coinciding with the base form, and the German plural, the distribution of a noun's collocates in the prepositional phrase usually involves some ten prepositions, and in the adjective-modified noun phrase, a noun may appear with a virtually unlimited number of adjectives. Hence,  the three structural contexts complement each other since the different properties of the explored structural patterns have varying repercussions for the effect of co-occurrence frequency on mixing patterns and are thus worth comparing.

Taken together, the three case studies embrace a range of understudied phenomena of bilingual speech, covering typical embedded-language islands (Chapters \ref{NP} and \ref{PP}) and internal embedded-language islands (Chapter \ref{PL}). The chapters have the following structure: The existing explanations for the scrutinised phenomena will be reviewed, including structural non-equivalence and frequency of co-occurrence, and complemented by further possible motivations, such as word repetition in discourse and morphophonological regularities. Systematic analysis and its results will be presented for each of the examined factors, their interplays will be evaluated statistically. The remainders of the chapters will summarise and discuss the results.

I will conclude this work by recapitulating its main findings, comparing predictors of the variation examined in each of the case studies and sketching some promising avenues for future research.

%Looking back on more than three decades of rigorous research on code-mixing/switching, we can certainly state that despite the intense research activity in the field, many questions about the organisation of bilingual speech have remained unanswered. Among them is the question of the competing motivations behind the distribution of code-mixing patterns in bilingual speech. This overarching question can be broken down to more specific questions, such as ``What are the psychologically plausible explanations of the observed code-mixing patterns?'' and ``How can these explanatory factors be adequately examined in a testable multilevel fashion?''. In search of explanations for the linguistic patterning in code-mixing, I consider such factors usage frequency, linguistic and discursive context as well as distinctive and overlapping features of the languages involved in code-mixing. 

%Code-mixing/switching, an inherently manifold phenomenon, is not an easy matter to grasp: any description thereof can be challenged for being incomplete and any explanation of its structure applicable to one particular situation can be refuted for being inadequate in another situation. 
