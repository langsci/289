\chapter{Previous research on the grammar of code-mixing}\label{CM}
This chapter presents an overview of current approaches to the grammar of code-switching/mixing, which is defined as the juxtaposition of two or more languages, or varieties, in discourse. After more than thirty-five years of thriving research, research in code-switching/mixing has developed itself into a well-established branch of linguistics, which has found its way into introductory linguistic textbooks. The study of code-switching/mixing has strong interdisciplinary links to diverse disciplines, such as sociolinguistics, conversation analysis, descriptive linguistics, language contact, language acquisition, linguistic anthropology, psycholinguistics and neurolinguistics. Given the substantial increase in the amount of literature devoted to code-switching/mixing, it is impossible to provide a complete overview of the field. Most comprehensive surveys of recent work are \citetitle{bullock-toribio} edited by Bullock and Toribio (\citeyear{bullock-toribio}) and the volume \citetitle{gardner-chloros_code-switching_2009} authored by Gardner-Chloros (\citeyear{gardner-chloros_code-switching_2009}). An apposite review of the earlier literature on grammatical aspects of code-mixing/switching is provided by  \citet[Chapter 1]{boumans-syntax-1998}. For this reason, I will only outline key modern advances in the field, which are relevant for the research presented in the following chapters. 

Before delving into theoretical aspects of code-mixing, the chapter will address an important terminological issue requiring disambiguation, the notorious dichotomy code-mixing versus code-switching. The chapter will thus open with a brief overview of the history of the terms and explain the division adopted in the present work. On tackling this issue, the chapter will proceed with the typology of code-mixing proposed by \citet[][]{muysken-bilingual-2000}. Recognising the variability inherent in code-mixing patterns, Muysken attempts an effective general classification and relates the identified types of code-mixing to structural, social, and psychological factors. A separate section will be devoted to social factors, which have been considered crucial to the emergence of  code-mixing patterns and their variability. As code-mixing in my Russian-German data is expected to be of the insertional type, I will present and discuss one of the most elaborate approaches to insertional code-mixing the Matrix Language Frame (MLF) model, authored by \citet[][]{myers-scotton-duelling-1993,myers-scotton-contact-2002}. multimorphemic and multiword insertions, a distinct type of insertional code-mixing, will be covered in a separate section. I will specifically look at these insertions from the perspective of the unit hypothesis and the ``conceptual unit'' hypothesis, articulated by \citet[][]{backus-evidence-1999,backus-units-2003}. The closing section of the chapter will focus on the ongoing controversy over the status of code-mixing and borrowing as distinct processes, and the proposed diagnostic criteria to distinguish between them. 

\section{A preliminary remark on terminology}
\label{sec:preliminary}
A juxtaposition of two or more languages in discourse has been studied from various perspectives and been referred to by different terms: code-alternation (\citealt{ extra-code-copying-1993,johanson-zum-1999, thomason-language-2001,migge-exploring-2013}), code-switching (\citealt{gumperz-social-1972, poplack-sometimes-1980,myers-scotton-duelling-1993,backus-two-1996}), code-mixing (\citealt{muysken-bilingual-2000,muhamedowa-untersuchung-2006}), language alternation (\citealt{auer-bilingual-1984, maschler-transition-1998}) and language-mixing (\citealt{pfaff-1979, backus-patterns-1992,lanza-language-2004}). Researchers often use more than one of these terms to refer to different phenomena of bilingual speech. For example, \citet{thomason-language-2001} employs the term \textit{code-switching}  for ``the use of material from two (or more) languages by a single speaker in the same conversation'' (p. 132), whereas the term \textit{code-alternation} is reserved for the diglossic use of languages by the same speaker (p. 136). Among the aforementioned terms, \textit {language-mixing} and \textit {code-switching} have the longest history. 

The use of the word \textit {language-mixing} in academic discourse goes back to \citet{haugen-1953-vol1}. He adopts this layman's term (cf. p. 58) to refer to the ``confusion of patterns'' observed in bilingual speech (p. 53). Alongside \textit{language-mixing}, Haugen speaks of \textit{switching} from one language to another. Although he does not provide a definition of the term, he relates it to the notion of \textit{switch}, which he defines as ``a clear break between the use of one language and the other'' \citep[65]{haugen-1953-vol1}. The term \textit{language-mixing}, as employed in \citet{haugen-1953-vol1}, can be interpreted as the umbrella term for both switching languages and borrowing. However, in the second volume of \citetitle{haugen-1953-vol2} (\citeyear{haugen-1953-vol2}) he abandons this term in favour of \textit{borrowing}. He explains his choice by the inadequacy of the term \textit{mixing} when applied to bilingual speech.\footnote{
Haugen writes, ``Mixture implies the creation of an entirely new entity and the disappearance of both constituents, or a jumbling of a more or less haphazard nature. But speakers have not been observed to draw freely from two languages at once, aside from abnormal cases. They may switch rapidly from one to the other, but at any given moment they are speaking only one, even if they resort to the other for assistance'' (\citeyear[362]{haugen-1953-vol2}).}

The first mention of the term \textit {code-switching} in relation to the juxtaposition of two or more languages is attributed to \citet{vogt-rev-1954}. In his review of Weinreich's \textit{Languages in Contact}, Vogt refers by this word to the alternate use of ``languages as well integrated systems, as codes'' (p. 81). However,  \citet{alvarez-caccamo-switching-1998} reports that the idea of  ``switching code'' was previously articulated by \citet{jakobson-preliminaries-1976}, who adapted this notion from information theory and related it to ``coexistent phonemic systems'' in a speaker's mind (p. 11). In the terrain of morphosyntax, it was \citet{haugen-bilingualism-1956}, again, who explicitly defined \textit{code-switching} as ``the alternate use of two languages'' (p. 40) and contrasted it with \textit{interference} and \textit{integration} (p. 50). According to \citet{alvarez-caccamo-switching-1998}, the work of the 1950s and early 1960s is characterised by a lack of consistency in the use of the terms \textit{code-switching} and \textit{language-mixing}. This situation is obviously the source of the terminological discrepancy observed in the field to date \citep[for overviews, see][]{ammon-code-switching-2004, ammon-code-switching-2005}.

In the present work, I will approach phenomena of bilingual speech using the typology proposed by \citet{auer-codeswitching-1999}. Auer distinguishes between the cases of \textit{code-switching} and \textit{code-mixing}, depending on the meaning participants ascribe to these events. Participants may perceive and interpret the juxtaposition of two codes, or languages, as locally meaningful, i.e., they may use this juxtaposition as a conversational device contributing to the local management of conversation. In this case  we deal with code-switching. In the case of code-mixing,  the alternate use of two languages is meaningful only in a global sense, that is, ``as a recurrent pattern, but not in each individual case'' \citep[467]{wodak-code-switching-2011}. The traditional division between these terms is based on whether switching occurs on the sentential level or below, it is thus distinguished between inter-sentential code-switching and intra-sentential code-mixing (\citealt[cf.][467]{wodak-code-switching-2011}, \citealt[70--73]{clyne-dynamics-2003}). The extent of the term \textit{code-switching}, as defined by Auer, coincides with that of the traditional definition in so far as code-switching in Auer's sense indeed occurs more often among full syntactic units. However, the meaning-based perspective taken by Auer is more pervasive and therefore  preferable than the traditional, form-based dichotomy because the former can account for transitional cases of code-switching, such as intra-sentential switching with local meaning attributed to it. Below, I will build on the opposition \textit{code-switching} versus \textit{code-mixing}\footnote{
Margaret Deuchar (p.c.) has pointed me to the fact that the term ``code-mixing'' has a negative connotation. This is probably one reason why this term is indeed infrequent and perhaps even intentionally avoided in studies originating in the USA and the UK. Another reason may be the tradition traced back to Haugen's explicit rejection of the term ``mixing''. Nevertheless, in order to maintain consistency with the established typologies of bilingual speech, I will use the term ``code-mixing'' without intending to evoke negative associations.}
to describe linguistic data and to review relevant theoretical approaches. Following \citet{muysken-bilingual-2000}, I will occasionally use the term ``switch'' to refer to a particular site where the juxtaposition of languages occurs.

\section{Typology of code-mixing}
Until recently researchers have not discriminated between different kinds of code-mixing. In studies that originated in the late 1990s, however, we can observe an interest in greater differentiation of phenomena pertaining to bilingual speech and their relation to each other \citep[cf.][]{auer-codeswitching-1999}. A most comprehensive typology of code-mixing, which draws on samples of bilingual speech encountered in various bi- and multilingual communities worldwide and involving different language constellations, has been proposed by \citet{muysken-code-switching-1997}. He further elaborates this model in the volume \textit{Bilingual speech: A typology of code-mixing} (\citeyear{muysken-bilingual-2000}). In this approach, the grammatical patterning in code-mixing yields three major structural types, which are brought into relation with typological, psycholinguistic and sociolinguistic factors. The latter circumstance makes this typology particularly appealing for analyses of bilingual speech and I will sketch it briefly below.

According to Muysken, code-mixing covers ``all cases where lexical items and grammatical features from two languages appear in one sentence'' (\citeyear[][1]{muysken-bilingual-2000}). Although this definition ignores the issue of meaning that participants can attribute to each case of juxtaposing codes in conversation, the scopes of this definition and the definition given by \citet{auer-codeswitching-1999} seem to roughly overlap (cf. section~\ref{sec:preliminary}). Yet, Muysken's definition is broader than Auer's in that the former includes cases of juxtaposition of codes within one sentence which may be perceived and interpreted by conversation participants as meaningful.

The typology distinguishes between three basic code-mixing processes: \textbf{insertion} of material from another language, \textbf{alternation} between structures of the involved languages and \textbf{congruent lexicalisation}. In \textbf{insertional code-mixing}, a lexical item, or an entire constituent, of language A is used in a structure of language B, where language B, called base language, usually maintains the frame for the overall clause. Insertions thus exhibit a nested B A B structure such that an insertion from language A is preceded and followed by fragments of language B that are grammatically related \citep[cf.][61--69]{muysken-bilingual-2000}. This type subsumes lexical borrowing, which may be regarded as a special case of insertion. Insertions are open-class words that function as complements rather than adjuncts, and they are usually subject to morphological integration. It is useful to distinguish between two kinds of insertion: minimal and maximal insertions. A minimal insertion occurs when a stem from another language combines with grammatical formatives of the base language, a maximal insertion refers to the case when both the inserted stem and the accompanying grammatical formatives come from the same language \citep[cf.][]{auer2014}, for instance:

\ea{\label{ex:1:1}} 
Swahili-English \citep[][80]{myers-scotton-duelling-1993}\\
\gll \textit{Leo} \textit{si-ku}-come \textit{na} books \textit{z-angru} {[\dots]}\\
	today \textsc{1sg.neg-pst.neg}-come with {} \textsc{cl10}-my\\
\glt `Today I didn't come with my books.' 
\z

\noindent In (\ref{ex:1:1}), the English verb \textit{come}, inserted in the Swahili matrix structure, receives Swahili verbal prefixes, but the nominal insertion \textit{book} bring in its English plural suffix. Whilst the minimal insertion \textit{come} undergoes morphological integration here, the plural \textit{books}, being a maximal insertion, is subject only to syntactic integration.

The next type of code-mixing is \textbf{alternation}. As the name suggests, the two languages alternate in a clause so that ``there is a true switch from one language to the other, involving both grammar and lexicon'' \citep[5]{muysken-bilingual-2000}. Thus, the speaker begins a sentence in one language and switches over to the other language. Alternation can occur at any point in the clause provided that the syntactic structures of the languages involved exhibit linear word order equivalence at that point \citep[cf.][114]{muysken-bilingual-2000}. For example, in (\ref{ex:1:2}) the alternation occurs after the Moroccan Arabic relativiser \textit{lli} `who'; at this point the syntax of Moroccan Arabic is equivalent to that of French.

\ea
\label{ex:1:2}
Moroccan Arabic-French \citep[311]{bentahila-davies-1983}\\
\gll \textit{kajn} \textit{bzzaf} \textit{djal} \textit{nna:} \textit{lli} ne font rien\\
	there are many people who \textsc{neg} do.\textsc{prs3pl} nothing\\
\glt `There are many people who do nothing.'
\z

\noindent In (\ref{ex:1:2}), French is maintained throughout the switched fragment: from the switch site to the end of the clause. In other words, there is no `return' to the language of the initial fragment after the switch. This case is one subtype of alternational code-mixing, its structure can be represented as A \dots{} B. In this case, the switched sequence involves several immediate constituents \citep[cf.][96]{muysken-bilingual-2000}. In (\ref{ex:1:2}), for instance, the French fragment encompasses a verb phrase and a noun phrase. The switched sequence may occasionally coincide with a whole subordinate clause, as shown by \citet[312]{pfaff-1979} and especially by \citet[196--200]{treffers-daller-mixing-1994}. In the extreme case, it may exceed the clause boundary, as is observed in the example below:

\ea
\label{ex:1:3}
English-Kashmiri \citep[228]{bhatt-1997}\\
\gll {\dots} and they also made \textit{gadɨ} \textit{kyazyiki} \textit{sanyi} \textit{bil-as} \textit{chi} \textit{bad} \textit{kʰof} \textit{kəran}.\\
	{} {} {} {} {} fish because our Bily-\textsc{dat} is very happy does\\
\glt `And they also made fish because our Billo (son's name) is very happy.' 
\z

\noindent The beginning of the switched fragment in (\ref{ex:1:3}), the Kashmiri \textit{gadɨ} `fish', may remind one of an insertion. We could assume that the Kashmiri noun may have been inserted into the matrix of the English sentence produced so far. However, an insertion would be possible here only if the noun were not followed by another syntactic unit in the same language, either a phrase or a subordinate clause. This pattern, although not ``clearly alternational'' \citep[102]{muysken-bilingual-2000}, still exhibits some of the characteristic features of alternation. Notably, the sentence in (\ref{ex:1:3}) has a distinct point of alternation from English to Kashmiri, and the switched sequence comprises a noun phrase and a subordinate clause. According to \citet[303]{auer2014} instances like (\ref{ex:1:3}) are quite common in bilingual corpora. As can be gleaned from examples (\ref{ex:1:2}) and (\ref{ex:1:3}), switched fragments pertaining to the discussed subtype of alternation are long and complex.

Another subtype of alternational code-mixing exhibits a non-nested structure A {\dots} B {\dots} A \citep[cf.][97--103]{muysken-bilingual-2000}. This means that a fragment of language B is preceded and followed by fragments of language A, but these latter fragments are not related grammatically. In this regard, the two Hungarian fragments in (\ref{ex:1:4}) -- \textit{szóval} `well' and \textit{hanem kottára} `but with notes' -- are not in syntactic relation, and we can thus analyse the example as an instance of alternational code-mixing.

\ea{\label{ex:1:4}}
German (dialect)-Hungarian \citep[260]{szabo-language-2010}\\
\gll \textit{szóval} net abgedreckt, \textit{hanem} \textit{kottá-ra.}\\
	well not printed but note.\textsc{sg-sub}\\
\glt `Well, not a printed [hymnbook], but one with notes.' 
\z 

\noindent Alternational code-mixing of this subtype commonly involves discourse markers, adverbs, adverbials, coordinated constituents, clefts and tags. Many of these elements usually occur at the periphery of the clause \citep[97--99]{muysken-bilingual-2000}. Both switching sites in (\ref{ex:1:4}) can be described as peripheral: First, the peripheral position of the Hungarian discourse marker \textit{szóval} `well' is unambiguous. Second, the Hungarian sequence \textit{hanem kottára} `but with notes' is a coordinated constituent, which means that the speaker could have finished her turn just before this sequence, i.e., after producing the participle \textit{abgedreckt} `printed'. Thus, the second Hungarian sequence is also located at the periphery.

At first sight, the two subtypes of alternational code-mixing, as exemplified in (\ref{ex:1:2}) and (\ref{ex:1:4}), may appear to be at odds with each other. In (\ref{ex:1:2}), the switch is placed within a complementiser phrase and the switched fragment is long and complex, whilst in (\ref{ex:1:4}), the switches are placed at the clause periphery and the switched fragments are short. Nevertheless, the two subtypes can be conflated because in both cases none of the involved languages is superimposed and the switched fragments are autonomous of each other. Furthermore, the two subtypes are understood as abstractions, and, of course, they can combine in bilingual speech.

One aspect of the outlined approach has been identified as problematic. According to \citet[][7]{muhamedowa-untersuchung-2006}, certain syntactic structures cannot be clearly assigned to one of the two types: alternation or insertion. She illustrates this point with the case of prepositional phrases, which are handled as instances of alternation in the typology. Yet, she contends that for mixed sentences containing prepositional phrases in the other language, the identification of the base language is non-ambiguous, and therefore, the other-language constituents may rather count as insertions (p. 8). Interestingly, in an earlier version of his model \citet{muysken-code-switching-1997} illustrates insertional code-mixing with the following example:

\ea{\label{ex:1:5}}
Spanish-English \citep[296]{pfaff-1979}\\
\gll Yo and-uv-e \textit{in} \textit{a} \textit{state} \textit{of} \textit{shock} pa dos día-s.\\
 \textsc{1sg} walk-\textsc{pst-1sg} {} {} {} {} {} for two day-\textsc{pl}\\
\glt `I walked in a state of shock for two days.' 
\z

\noindent Here, the English prepositional phrase \textit{in a state of shock} is analysed as an insertion. According to the later version of the model, this example would obviously be regarded as an instance of alternation. Generally speaking, the case of prepositional phrases shows that insertional and alternational code-mixing should be considered as two non-discrete categories \citep[cf.][95]{backus-two-1996}.

The third type of code-mixing is \textbf{congruent lexicalisation}. \citet[3--4]{muysken-bilingual-2000} speaks of congruent lexicalisation when the grammatical structure shared by the languages in contact is filled by lexical items belonging to either language. The involved languages have to therefore exhibit a great extent of both linear and categorical equivalence such that lexicalisation is made possible. Congruent lexicalisation is also likely when the languages at play manifest a low degree of linear equivalence, but a similar vocabulary including homophonous words, which can trigger code-mixing. \citet[][123]{muysken-bilingual-2000} draws on Dutch-English bilingual speech to illustrate this pattern. Although Dutch and English exhibit some divergence in word order patterns, their vocabularies overlap to a considerable degree. In this case, homophonous words function as triggers for code-mixing. This scenario can arguably unfold in a situation of contact between any two closely related languages as they often share a large stock of words. Let me illustrate congruent lexicalisation which results from contact between two closely related languages Russian and Ukrainian:

\ea{\label{ex:1:6}}
Ukrainian-Russian \citep{vahtin-novye-2003}\\
\gll To ž oce pered \textit{prazdnik-om} \textit{plit-k-u} pomy-l-a.\\
	\textsc{ptcl} \textsc{ptcl} \textsc{ptcl} before holiday-\textsc{instr.sg.m} cooker-\textsc{dim}-\textsc{acc.sg.f} wash-\textsc{pst-sg.f}\\
\glt `Well then before the holiday [I] washed the cooker.' 
\z

\noindent Russian and Ukrainian not only have a similar vocabulary like English and Dutch, but also share a large part of grammatical structure. Although both languages are traditionally described as free-word order languages, the constituent order observed in (\ref{ex:1:6}) is only relatively free: it serves to express  topic-focus relations. The topic generally precedes the focus in both Russian and Ukrainian. As the grammars of the contact languages require, the topic in (\ref{ex:1:6}), realised by the mixed phrase \textit{pered prazdnikom} 
`before the holiday', is fronted. The adpositional phrase in question is headed by the Ukrainian preposition \textit{pered}  `before', which projects the instrumental case on the Russian singular noun \textit{prazdnik} `holiday'.\footnote{Juliane Besters-Dilger (p.c.) has drawn my attention to the fact that the use of the preposition \textit{pered} in the temporal meaning is far less common in Ukrainian than in Russian and the preposition could thus be analysed as Russian rather than Ukrainian. I thank her for this observation.} The remainder of the sentence \textit{plitku pomyla} `washed the cooker' expresses the focus. As both languages use the same syntactic pattern to code topic-focus relations, it is impossible to say whether the constituent order in (\ref{ex:1:6}) is Russian or Ukrainian. The word order in the adpositional phrase is again identical for both languages since both heavily rely on prepositions. Crucially, not only linear but also categorical equivalence plays a important role in the examined instance. Notably, patterns of case assignment observed in the prepositional and verb phrases are the same: the preposition \textit{pered} `before' assigns its complement the instrumental case in both languages, and the Russian verb \textit{pomyt'} `wash', just like its Ukrainian counterpart \textit{pomyti}, assigns its complement the accusative case. Following \citet[129]{muysken-bilingual-2000}, the switch within the prepositional phrase \textit{pered prazdnikom} is conditioned by the structural equivalence in the prepositional phrase and the noun phrase. He asserts that structural equivalence leads to multi- and non-constituent mixing, another feature of congruent lexicalisation \citep[129]{muysken-bilingual-2000}. 

Let me now turn to the lexical correspondences observed in (\ref{ex:1:6}). Here, the Russian \textit{plitku} `cooker', used in the accusative case, is virtually identical to the corresponding Ukrainian form \textit{plytku}. The two words differ in the stem vowel -- Russian /i/ versus Ukrainian /ɪ/ -- and the preceding consonant, i.e., Russian /lʲ/ versus Ukrainian /l/. The verb form \textit{pomyla} is Ukrainian and deviates from its Russian equivalent only in the vowels, namely, the Ukrainian [pɔˈmɪɫa] contrasts the Russian [pʌˈmɨɫɑ], or [pʌˈmɨɫə]. We can thus consider the words \textit{plytku} and \textit{pomyla} as homophonous diamorphs. They illustrate the dependence of congruent lexicalisation on a common vocabulary stock. With the exception of the Ukrainian discourse marker \textit{to ž oce} `well then', the only lexical item in (\ref{ex:1:6}) that contributes to lexical divergence is \textit{prazdnik} `holiday', its Ukrainian equivalent is the word \textit{svjato}. Since the instrumental marking on the noun \textit{prazdnik} is also identical in both languages, we may assert that the structural equivalence  maintains lexical insertion here. Crucially, structural congruence as such does not have to be total. For example, congruent lexicalisation may be observed when mixing occurs between languages such as Spanish and English. In this case congruent lexicalisation results from partial congruence \citep[cf.][6]{muysken-bilingual-2000}.

A important generalisation concerning congruent lexicalisation is that categorical and linear equivalence lead to a situation in which mixing is syntactically unconstrained. Any category can be switched, and even word-internal switching is possible. For this reason Muysken views congruent lexicalisation as akin to style shifting and language variation, such as observed between a standard and a dialect (\citeyear[127--128]{muysken-bilingual-2000}). We can thus conclude that congruent lexicalisation depends on structural equivalence, lexical correspondence, or both, as in (\ref{ex:1:6}), and it is distinguished by non-nested \textit{a b a} structure \citep[129]{muysken-bilingual-2000}.

The comprehensive character of the typology makes it an attractive tool for investigating the linguistic patterning of bilingual speech. Analysis of code-mixing aimed at probing into linguistic variation and change as well as their correlates draw on the mixing types (and their subtypes) to describe patterns in naturally occurring bilingual speech. For instance, \citet[][]{chen-styling} identifies distinctive code-mixing styles in Hong Kong, with one style allowing for insertion and alternation, and the other involving only insertion. Another example is the work by Goria (\citeyear[][]{goria-inglese, goria-road}), in which he reports differing alternational patterns in Gibraltar's Spanish-English bilingual speech and attributes them to three generational cohorts of bilingual speakers and ultimately to an ongoing language shift from Spanish to English. Although these studies demonstrate the usefulness of this approach, some of its caveats and intricacies should be mentioned. First, all three code-mixing types can co-occur in a corpus of bilingual speech \citep[229]{muysken-bilingual-2000}. It is therefore necessary to determine the dominant code-mixing type in the corpus. Second, ``intermediate cases may exist'' \citep[229]{muysken-bilingual-2000}. As discussed above, prepositional phrases form a category with an intermediate status. Adverbials, which count as indices of alternational mixing, may in principle be analysed as inserted into a base language. %Considering these limitations, an analysis drawing on the outlined typology should not only subject each individual instance of mixing to thorough scrutiny, but also give due consideration to the general tendencies in the sample.

Another merit of the outlined typology is that it relates the identified code-mixing types to linguistic, or typological, and extralinguistic, or socio- and psycholinguistic, factors \citep[][221--249]{muysken-bilingual-2000}. The evaluated correlations allow predictions about the predominant code-mixing type in a specific community given the contact languages' typological profiles, the speakers' language dominance patterns, the specifics of the sociolinguistic situation and other factors. Considering Muysken's conclusions, it is possible to assume, for instance, that the predominant code-mixing type in the bilingual speech of Russian Germans in Germany, the subject matter of this book, will be insertion. At the grammatical level, Russian and German, being fusional languages, exhibit a high degree of typological proximity, but their word order patterns and core vocabularies are not similar enough for congruent lexicalisation to emerge. The social conditions of the examined situation, namely, repatriation after a language shift to Russian (for details, see \ref{RuDe}), also favour insertion. Finally, the speakers' bilingual proficiency in Russian and German allows for intensive code-mixing. The three groups of factors work jointly and are often included in analyses of bilingual speech as individual independent variables \citep[cf.][]{muysken-etal96}. At the same time, many researchers have emphasised that social factors take priority over other factors in language contact. In what follows I will discuss this issue and illustrate the role of social factors in code-mixing by providing examples from the literature.

\section{Social factors in code-mixing}
This section begins with an outline of the existing attempts to systematise  social factors influencing the linguistic structure of bilingual speech and then showcases the factors that are particularly relevant to the linguistic community and the speakers whose bilingual speech is analysed in this book. Before I turn to these topics, I discuss the claim that among other factors, social factors play a major role in code-mixing, and in language contact in general.
 
\subsection{Social factors versus structural and psychological factors}{\label{social-factors}}
Researchers who relate differing outcomes of language contact to distinct types of sociohistorical contexts in which language contact occurs, particularly Sarah (Sally) Thomason, have strongly advocated for the view that ``when social factors and linguistic factors might be expected to produce opposite results in a language contact situation, the social factors will be the primary determinants of the linguistic outcome'' (\citeyear[][42]{thomason-08}). This position is not alien to \citet{muysken-bilingual-2000}. In a treatment of structural factors, he acknowledges the role of categorial equivalence between linguistic structures of the contact languages, but emphasises that historical and sociolinguistic factors are more reliable for determining the likelihood of a specific mixing pattern in a given situation because ``categorial equivalence is not a purely objective notion'' (p. 247). Although quite plausible, this reasoning does not elucidate the nature of the relation between social factors and congruence, even when the latter is regarded as a subjective phenomenon. An account of this relation may build on the view that identification of equivalence, or what \citet[][]{weinreich-53} labelled as ``interlingual identification'', is bound to a single individual \citep[see the discussion in][37]{gardner-chloros_code-switching_2009} and is intrinsically embedded in social interaction. Since interlingual identification, being an effect of a more general process of similarity detection, depends on an individual's previous linguistic and interactional experience \citep[for more details, see][]{hakimov-17,hakimov-backus-20-intro}, the outcomes of this process are diverse and indeterminate. Yet, it is more likely than not that they are constrained by the individual's interactions in social networks, or a larger speech community.\footnote{Pertinent mechanisms at play are accommodation \citep{giles-80}, and focusing \citep{lepage-tabouretkeller}.} Innovative usage patterns attributable to individual unconventional interlingual identifications may either go unnoticed in interactions, or be sanctioned by the speaker's interaction partners; alternatively, they may be perceived and adopted by the interaction partners and gradually diffuse in the speech community. In other words, the emergence of innovative usage patterns resulting from subjective interlingual identifications, or individually established congruence, is inseparable from the interactions in the social networks and the community. Innovations may also spread to neighboring bilingual communities and give rise to local norms.

A recent example of such a locally emerged innovation is provided by \citet[][]{bullock-serigos-toribio}. The authors report an unconventional use of the Spanish verb \textit{agarrar}, meaning `grab' or `grasp', in a corpus of Spanish spoken in Texas. Utilising variationist and corpus-linguistic methodology, they demonstrate that the combinations of this verb with abstract nouns such as \textit{ayuda} `help', or \textit{experiencia} `experience' are calqued on English \textit{get}+NP support verb constructions such as \textit{get help}. Crucially, the reported usage patterns of the verb \textit{agarrar} are not registered to the same degree elsewhere. It is highly plausible that local norms affect code-mixing to a similar extent as they influence linguistic transfer.

Of the psycholinguistic factors at work in code-mixing, \citet[][224--227]{muysken-bilingual-2000} pays particular attention to bilingual proficiency and cites several studies probing into the relationship between bilingual proficiency and the rate and type of code-mixing. He concludes that although most studies report positive correlations between bilingual proficiency and the extent of code-mixing, the relation is complex and often mediated by social factors. Among them, he mentions network membership, prestige and generational membership in a migrant community (for details, see below).

The interplay between psycholinguistic aspects of multilingualism and the social nature of linguistic boundaries is evaluated by \citet{law-14}. Although he does not specifically discuss code-mixing, he emphasises the social nature of language separation in the bilingual mind, asserting that ``[a] central process in `language contact' is the merging and separation of different elements of linguistic systems in the bilingual mind, and that separation is intimately social and extremely variable and dynamic, not only from person to person, but within a single individual’s own mind over time'' (p. 162). When applied to bilingual speech, this view assumes that selection and use of linguistic structures of one language in the discourse framed by another language depends on the individual's ideological awareness of distinction between mixed and unmixed speech as well as the interactional context permitting language mixing. Empirical evidence in support of this position is indirect but encouraging. In two language comprehension experiments, \citet{adamou-shen} found that processing costs in mixed utterances are reduced if code-mixing is socially acceptable and frequent. In other words, routine use of elements of one language in juxtaposition with elements of another language have ramifications for patterns of activation of linguistic representations in the mental grammar/lexicon and its overall organisation.

All in all, the view that social factors are primary determinants of linguistic patterning in bilingual speech has become widely accepted. Yet, work is still lacking that systematises and evaluates different aspects of social behaviour and power relations that have been linked to various settings in which code-mixing takes place. Existing classifications of the social factors affecting code-mixing include the proposals by \citet{muysken-bilingual-2000} and \citet{gardner-chloros_code-switching_2009}. Although both approaches are hardly exhaustive, they are genuinely useful. I will  describe and contrast them below and then complement the outline by a presentation of the individual factors relevant to the studies reported in the subsequent chapters.

\subsection{Types of social factors}
In his attempt to categorise social factors favouring the emergence of code-mixing, \citet[][222--223]{muysken-bilingual-2000} puts the contexts in which code-mixing occurs center stage. Although we find no explicit reference to social factors as such in this approach, contexts in which communication takes place are socially determined and are thus interpretable in terms of social factors. Muysken allocates social contexts to one of three analytical levels: the macro, the meso, and the micro level. The macro level draws on aspects of social and political structure that are characteristic of the bilingual situation on a large scale. The bilingual settings at this level include frontier regions between languages, clusters of multilingual tribal groups with reciprocal bi- and multilingualism, dialect/standard language relations, minority language islands, bilingualism of native elites, colonial and post-colonial settings, and migrant communities. The meso level in \citeauthor{muysken-bilingual-2000}'s catalogue describes bilingual communities in terms of their sociolinguistic profiles. The list encompasses aspects such as ``the degree of acceptance of code-mixing in the community, attitudes towards bilingualism, structures of linguistic domination, whether it is a transplanted or endogenous bilingual community, the distribution of patterns of language use, including bilingual speech across generations'' (p. 222). The interactional setting corresponds to the micro-level. Among the investigated contexts at this level we find peer group and family interactions, institutional interaction in class rooms and in public-authority bodies, marketplace transactions, and exploratory conversations between relative strangers. Muysken emphasises that the proposed list of contexts is incomplete. An apparent consequence of this approach is that the bilingual individual's linguistic behaviour is viewed as a product of contexts located at different analytical levels. In an account of code-mixing, the analyst's task is thus to identify these contexts, to relate them to each other and eventually to the observed mixing patterns.

Muysken further links the various contexts and other social aspects of multilingualism such as, for instance, attitudes towards bilingualism and the existence of strong linguistic norms to specific structural types of code-mixing, namely, insertion, alternation and congruent lexicalisation. As a dominant code-mixing pattern, insertion is common in (post-)colonial settings and in immigrant communities in the first and intermediate generations (see below). The language providing insertions is usually associated with political power, it is the language of the new country in a situation of immigration and the language of the (former) metropolitan country in (post-)colonial settings. According to Muysken, alternational code-mixing is common among immigrants of the second and following generations, and is also typical in communities characterised by strong norms concerning linguistic behaviour (p. 249). Finally, congruent lexicalisation is facilitated by loose linguistic norms, a balance between the involved languages, and structural parallels.

The reference points of the typology of social factors proposed by \citet[][42--43]{gardner-chloros_code-switching_2009} are the speaker and the interaction in which they are involved. Factors of the first type include those that are situated beyond the speaker and the specific interactional context; these are factors that ``affect all the speakers of the relevant varieties in a particular community, e.g., economic `market' forces such as those described by \citet{bourdieu-91}, prestige and covert prestige \citep{labov-72,trudgill-74}, power relations, and the  associations of each variety with a particular context or way of life \citep{gal-79}'' (p. 42). Factors of the second type refer to the bilingual individual, also as a member of social sub-groups. These factors pertain to such aspects of social structure as social networks and relationships, linguistic attitudes and language ideologies, perception of self and of  others. Also included in this group is proficiency in each variety. According to \citeauthor{gardner-chloros_code-switching_2009}, the individual's proficiency (competence in her terminology) is essentially ``a product of their (reasonably permanent) psycholinguistic make-up'', but it has sociolinguistic implications because it is influenced by social factors such as age, network, identity, etc. Although this decision is justified theoretically, in practice, the task of assessing the speaker's proficiency in a variety would inevitably require collecting research data through experimental work (e.g., the application of vocabulary-based proficiency tests) in order to supplement traditional fieldwork data.\footnote{Another approach to this issue may be illustrated by a study carried out by  \citet{muysken-etal96}, in which the authors refrain from collecting naturally occurring spontaneous conversations as the basis for the analysis and draw on a range of other data instead, including the recording of bilingual parent-child reading sessions.} The final bundle of factors encompasses factors operating within the interactional context. In this case, bilingual speakers employ the juxtaposition of languages, or varieties, in conversation as a contextualisation cue \citep[see e.g., the papers in][]{auer-98}. With the distinction between code-switching and code-mixing in view, we may assert that the factors at this level pertain to code-switching, i.e. a situation in which speakers perceive and interpret the juxtaposition of codes in a specific instance as a conversational device, but not to code-mixing, which the speakers consider meaningful only in the global sense but not in each individual case. 

The classifications of social factors by \citet[][222--223]{muysken-bilingual-2000} and \citet[][42--43]{gardner-chloros_code-switching_2009} exhibit a considerable overlap; each distinguishes between three types of factors, or three levels of analysis, respectively, and the first and second order categories in both approaches coincide to a great extent. The third order category is conceptualised differently in each case: In Muysken's classification it refers to the interactional setting as the global context in which the interaction takes place, whereas Gardner-Chloros' conversational factors pertain to local contexts in which two, or more, languages are meaningfully juxtaposed in discourse. Another discrepancy between the two approaches lies in the fact that alongside the social dimension of language contact, Muysken introduces its historical dimension, namely the duration of contact, as a fourth level. A more substantial difference is in the object of analysis: While Muysken's concern is with contexts allowing for code-mixing, Gardner-Chloros' focus is on factors. A specific characteristic of Gardner-Chloros' typology is the inclusion of factors pertaining to the organisation of conversation, i.e., the use of codes as a resource for managing interaction. Obviously, this level of analysis relates to switching codes as a bilingual practice, but not to language mixing. As the corpus analysed in the present work contains only few instances in which the speakers employ the juxtaposition of codes as a contextualisation cue, e.g., for quotation, the dominant pattern in the corpus is language mixing. I thus abstain from giving a detailed outline of factors pertaining to code-switching in the present overview, while I acknowledge the possibility that code-switching, and particularly its directionality, may influence patterns of mixing, should both types of code juxtaposition be accepted in a bilingual community. Below, I will showcase factors belonging to the first two types.

Of the factors independent of the bilingual individual, i.e., those operating at the macro level, patterns of sociopolitical dominance appear to be crucial because they influence the direction of insertion in code-mixing \citep[][224]{muysken-bilingual-2000}. Among the factors pertaining to the bilingual individual, of particular importance to the sampled group of speakers is generational membership, which is often related to the speakers' bilingual proficiency.

Differing patterns of sociopolitical dominance in social contacts may produce distinct patterns of linguistic structure. For instance, mixing patterns in bilingual speech may be inverted if an imbalance in power/status is reversed. Such is the case in the speech of Russian Germans after the shift to Russian and following their repatriation to Germany (for details, see \ref{RuDe}) when compared to the speech of Siberia's Russian Germans who have not shifted to Russian. Both groups employ the same strategies of verb integration in bilingual speech: the verb from the other language is adapted by adding either phonological or morphological material between the stem and the grammatical suffix, (\ref{ex:1:7a}, \ref{ex:1:8a}), or it is inserted as a frozen form (\ref{ex:1:7b}, \ref{ex:1:8b}). The data from Siberia provided by \citet[][]{blankenhorn} include the following examples:

\ea
German (dialect)-Russian \citep[][103,93]{blankenhorn}\\
\ea{\label{ex:1:7a}}
\gll [\dots] die hen sich so ge-wunder-t, dass ah mir scho all da \textit{vstreča}-i-t hen, un \textit{provoža}-i-t [\dots] \\
    {} they have.\textsc{3pl} \textsc{refl} so \textsc{ptcp}-wonder-\textsc{ptcp} \textsc{comp} also we already all here 
    meet-\textsc{sf-ptcp} have.\textsc{3pl} and see\_off-\textsc{sf-ptcp}\\
\glt `\dots they were so surprised that we also met people here and saw them off\dots'
\ex{\label{ex:1:7b}}
\gll [\dots] \textit{nu} \textit{vot} \textit{tAk} \textit{vot}, die hen/ die leit hen uns \textit{vytjanu-l-i}.\\
	{} \textsc{ptcl} \textsc{ptcl} \textsc{ptcl} \textsc{ptcl} they have.\textsc{3pl} \textsc{det.def.pl} people have\textsc{3pl} us pull\_through-\textsc{pst-pl}\\
\glt `Well that's just how it is. The people pulled us through.'
\z
\z

\noindent My data from Germany show the mirror image. While the Germans sampled in Blankenhorn's \citeyear[][]{blankenhorn} study insert verbs from Russian, the politically dominant language in that case, the speakers recorded for the present research in Germany do the opposite: they insert verbs from German, the ``new'' majority language, for instance:

\ea
Russian-German (own field data; for details, see \ref{RuDe})\\
\ea{\label{ex:1:8a}}
\gll prosto referat skaž-u xoč-u \textit{halt}-ova-t’.\\
	simply presentation[\textsc{sg.akk}] say.\textsc{pfv.prs-1sg} want.\textsc{prs-1sg}   make-\textsc{sf-inf}\\
\glt `I'll just say I want to make an oral presentation.'
\ex{\label{ex:1:8b}}
\gll \dots čë za nedelj-u \textit{passier-t}\\
    what during week-\textsc{acc.sg.f} happen-\textsc{ptcp}\\
\glt `\dots what happened during the week.'
\z
\z

\noindent Comparisons of bilingual speech emerging in contexts with opposite dominance relations between the contact languages are scarce, but \citet[][223--224]{muysken-bilingual-2000} provides one example of the case. He compares the patterns of verb integration in Central American English Creole \citep[based  on][]{herzfeld-80, herzfeld-83} and Mexican American Spanish \citep[based on][]{pfaff-constraints-1979} to show that the turnabout of the linguistic patterns can reflect the reversal of sociopolitical dominance. These observations lead to a more general conclusion that the structural patterns found in bilingual speech may mirror asymmetric quality of contact due to an imbalance in status between social groups.

The next factor pertinent to the group whose speech is the subject of this book is generational membership. Several studies have provided evidence that structural patterns of code-mixing may be related to generational differences \citep[for a review, see][224--227]{muysken-bilingual-2000}. To illustrate this factor, I will draw on two studies, one conducted in a (post-)colonial setting, the other carried out in the context of migration. 

The study by \citet{goria-road} analyses mixing patterns in Spanish-English bilingual speech of Gibraltar\footnote{The political status of Gibraltar is disputed. The description of this British dependent territory as a colony has been criticised by the UK authorities.} across three generations of speakers: speakers with age over 60 years, speakers between the ages 30 and 60 years, and speakers younger than 30 years. They are labelled as `elderly', `adult' and `young' speakers. The examined corpus of bilingual speech contains instances of insertional mixing as well as as tokens of congruent lexicalisation, but the dominant and most variable mixing pattern is alternation. Specifically, Goria looks at the patterns of clause-peripheral code-mixing and its distribution across the three generations. The results indicate that the factor ``speaker generation'' is an important predictor of the language of the extra-clausal constituent. For example, the use of Spanish conjunctions and complementisers linking English clauses comprises only six per cent of all cases in the speech of the elderly group, but amounts to 23 and 32 per cent in the speech of the adult and the young group, respectively. The author interprets these generational differences in mixing in terms of an ongoing shift from Spanish to English as a sociopolitically dominant language.

In his analysis of code-mixing in the Turkish-speaking community in The Netherlands, \citet[][387--391]{backus-two-1996} reports a correlation between generational membership in the migrant community and a specific type of code-mixing. He observes that the dominant pattern in the speech of first-generation immigrants is the insertion of Dutch content words and their morphosyntactic integration into Turkish (I have referred to this pattern as minimal insertion above). The vernacular of the intermediate generation speakers -- these immigrants arrived in the Netherlands when they were between 5 and 12 years old -- has as much insertion as alternation, and insertional mixing is highly varied as it includes both minimal and maximal insertions, the latter being fully-fledged Dutch constituents in otherwise Turkish sentences. Finally, the speech of the second-generation immigrants is characterised by alternational mixing. An overview of these findings is given in Table (\ref{tab:1:1}). As evident from the table, the first and the intermediate generation share the same propensity to use Dutch words in otherwise Turkish sentences, whereas the second generation also uses Turkish words in Dutch sentences. Furthermore, Backus links the generation-specific mixing patterns to the generations' language choice preferences, with a gradual shift from Turkish to Dutch. In his \citeyear[][]{backus06} publication, he cites several studies documenting the same pattern of intergenerational variation in other Turkish-speaking communities across Europe and concludes that this pattern may be very general. 

\begin{table}
\begin{tabular}{lcccccc} 
\lsptoprule
Generation & \multicolumn{3}{c}{Type of code-mixing} & \multicolumn{3}{c}{Base language in code-mixing}\\
\cmidrule(lr){2-4} \cmidrule(lr){5-7}
& insertion & both & alternation & Turkish & both & Dutch \\
\midrule
First & {\langscicheckmark} & &  & {\langscicheckmark} & & \\
Intermediate & & {\langscicheckmark} &  & {\langscicheckmark} & & \\
Second & & & {\langscicheckmark} & & {\langscicheckmark} & \\
\lspbottomrule
\end{tabular}
\caption{Distribution of main types of code-mixing, and base language in code-mixing across first, intermediate, and second generations in Turkish-Dutch code-mixing data \citep[adapted from][702]{backus06}.}
\label{tab:1:1}
\end{table}

In view of the Russian-German community in Germany, whose mixing patterns are reported in the subsequent chapters of this book, we can expect that the pattern of intergenerational variation in their speech will largely coincide with the pattern reported by Backus. It is important to emphasise however that the described Turkish community and the Russian-German community in Germany differ in their official status. Unlike the Turkish immigrants in the Netherlands, Germans from the Soviet Union and its successor states are repatriates to Germany. Yet, as will be demonstrated below (see \ref{RuDe}), the patterns of their language choice preference, considering their shift to Russian in the 1970s and 1980s, are comparable with those of immigrants. Against this background, and in view of the goal to explore patterns of insertional code-mixing, it is reasonable to assume that the speech of intermediate-generation Russian-German repatriates will provide diverse loci of variation in insertional code-mixing, including minimal and maximal insertions in the same language, and will hence suit the envisaged goal.

In addition to the sociolinguistic perspective, patterns of insertional code-mixing have been approached along the lines of structural analysis \citep[e.g.,][]{halmari-government-1997,boumans-syntax-1998,verschik08}, but one of the most elaborate structural approaches to insertional code-mixing is the Matrix Language Frame model \citep[cf.][363]{muysken-code-switching-1997}. To this, I turn next.

\section{The Matrix Language Frame model and its extensions}{\label{MLF}}
The Matrix Language Frame model was proposed and has been further developed by Carol Myers-Scotton  (\citeyear{myers-scotton-duelling-1993,myers-scotton-contact-2002}). Since its first formulation, this model has become the object of a heated controversy. Some scholars have successfully tested the model on various language pairs (see \citealt[]{haust-codeswitching-1995}, for Mandinka, Wolof-English;  \citealt[]{backus-two-1996}, for Turkish-Dutch;  \citealt[]{hlavac-second-generation-2003}, for Croatian-English;
\citealt[][]{amuzu-composite-2010}, for Ewe-English) or applied it to other language contact phenomena, such as creole formation, long-standing language contact, child bilingualism, and adult second language acquisition \citep[cf.][]{myers-scotton-testing-2000}. Other researchers, however, have criticised some of the model's assumptions (\citealt{meechan-orphan-1995}; \citealt{halmari-government-1997};   \citealt{jacobson-codeswitching-1998};  \citealt{boumans-syntax-1998}; \citealt{auer-embedded-2005};  \citealt{muhamedowa-untersuchung-2006};  \citealt{bullock-toribio-chang};  \citealt{zabrodskaja-evaluating-2009}). From the beginning, Myers-Scotton has developed and modified the original model further. I will therefore discuss the proposed models in the chronological order, i.e., the Matrix Language Frame model first and then its extensions: the Abstract Level model and the 4-M model.

\subsection {The Matrix Language Frame model}
As outlined above, in the case of insertional code-mixing an asymmetry is observed between the languages involved because only one language is responsible for providing the frame for the sentence. This language is called the \textbf{matrix language (ML)}, whereas the other language is referred to as the \textbf{embedded language (EL)}. According to Myers-Scotton (\citeyear{myers-scotton-duelling-1993}), this terminology goes back to \citet{joshi85}. His approach to code-switching builds on the observation that speakers and hearers are capable of identifying the language ``the mixed sentence is `coming from' '' (\citealt{joshi85}, quoted from \citealt[35]{myers-scotton-duelling-1993}). In contrast to the perceptual view taken, both Joshi and Myers-Scotton adopt a structural perspective on the matrix language-embedded language asymmetry \citep[cf.][35--37]{myers-scotton-duelling-1993}. As this asymmetry is the crux of the outlined model, the question of determining the matrix language is essential to this approach. 

According to \citet[68]{myers-scotton-duelling-1993}, more morphemes come from the matrix language than from the embedded language, where morphemes are counted in a discourse sample and cultural borrowings are excluded from the counts. This criterion for matrix language identification is not unproblematic. For example, \citet[19]{muhamedowa-untersuchung-2006} claims that her bilingual data contain lengthy monolingual passages which intervene with instances of code-mixing and code-switching (\citealt[102]{haust-codeswitching-1995}; \citealt[154]{boumans-syntax-1998}; and \citealt[196]{hlavac-second-generation-2003}, argue a similar point). That is, the discourse-dominant language may not coincide with the matrix language of a clause \citep[cf.][]{heller-conversation-1988}. In this circumstance, adequate morpheme counts are infeasible. In her later work, \citet[237]{milroy-lexically-1995} mentions two further criteria for matrix language definition: First, the matrix language is the unmarked choice in bilingual communication, one of the functions of which is solidarity building. Second, speakers' self-reports on which language is the matrix language are a reliable indication of the matrix language. In spite of the two suggested criteria, Myers-Scotton's analysis of bilingual sentences is virtually always based on solely structural criteria, related to another important premise of the matrix language frame model, i.e., the hierarchy between content and system morphemes.

Myers-Scotton differentiates between content and system morphemes. Prototypical system morphemes subsume function words and inflectional affixes, whereas prototypical content morphemes include verb and noun stems. A more detailed morpheme classification is based on such discreet categories as [±Quantification], [±\textit{θ}-role assigner], [±\textit{θ}-role receiver] and their role at the discourse level. System morphemes are characterised as [+Quantifier]. Syntactic categories with the feature [-Quantifier] are further classified with regard to their potential to assign or receive theta-roles. For example, nouns, pronouns, adjectives and adverbs derived from adjectives are theta-role receivers and thus content morphemes, but dummy pronominals \textit{there} and \textit{it} are not theta-role receivers and are therefore system morphemes. Because some syntactic categories assign and receive theta-roles, depending on the specific items that belong to these categories, they can function as either content or system morphemes. Prepositions are one of such categories; for instance, the English preposition \textit{in} and its French counterpart \textit{dans} are content morphemes because they assign both theta-roles and case, whereas the English \textit{of} and the French \textit{de}, marking genitive objects, are system morphemes as they assign only case \citep[cf.][98--102]{myers-scotton-duelling-1993}. Among verbs, which are typical theta-role assigners, the copula and the English \textit{do} in \textit{do}-constructions are considered system morphemes. The system morpheme versus content morpheme hierarchy and the matrix language versus embedded language hierarchy are related because system morphemes participate in constituent frame formation and can be controlled by only one language at one point in time \citep[235]{milroy-lexically-1995}. The two hierarchies are at the core of the Matrix Language Frame model.

%Boumans (1998) provides a detailed account of matrix language identification on both  the sentential and the phrasal level.

The model draws builds on several hypotheses. The \textbf{matrix language hypothesis} seeks to explain the structure of mixed, or ML + EL,  constituents in code-mixing, it says ``[...] the ML provides the morphosyntactic frame of ML + EL constituents'' \citep[82]{myers-scotton-duelling-1993}. This hypothesis determines the Morpheme Order and the System Morpheme Principles, namely:
\begin{itemize}
\item The Morpheme Order Principle: Morphemes in mixed constituents are ordered according to the ML. 
\item The System Morpheme Principle: Syntactically relevant system morphemes in mixed constituents come from the ML. \citep[cf.][239]{milroy-lexically-1995} 
\end{itemize}

\noindent A system morpheme is syntactically relevant if it is involved in agreement relations external to its head constituent. In (\ref{ex:1:9}), for instance, the matrix-language (i.e., Croatian) system morpheme \textit{-u} expressing the accusative case is considered  syntactically relevant because it is required by the head of the prepositional phrase and not the noun as the head of the noun phrase.

\ea{\label{ex:1:9}}
Croatian-English \citep[115]{hlavac-second-generation-2003}\\
\gll {\dots} sad ć-e ić u Hrvatsk-u za ov-u treć-u \textit{term}-u [{\dots}]\\
	{} now \textsc{fut-3sg} go.\textsc{inf} to Croatia-\textsc{acc.sg.f} for this-\textsc{acc.sg.f} third-\textsc{acc.sg.f} \phantom{term}-\textsc{acc.sg.f}\\
\glt `\dots he will be going to Croatia for this third term [\dots]'
\z

According to \citet[110]{myers-scotton-duelling-1993}, the System Morpheme Principle is maintained if any of three possible strategies is employed: The first, prototypical, case is the occurrence of mixed constituents, with system morphemes coming from the matrix language. System morphemes may also come from both languages simultaneously, so that \textbf{double morphology} emerges. This is the second strategy. As such, morphological doublets are only possible when the system morphemes of the embedded language do not have relations external to their heads. As a third strategy, EL content morphemes may appear as \textbf{bare forms}. \citet[112]{myers-scotton-duelling-1993} asserts that bare forms are produced if an embedded-language system morpheme and the corresponding matrix-language system morpheme are incongruent. The outlined constraints are integrated into a production model, which is largely based on the work by Levelt (\citeyear{levelt89}, quoted in \citealt[][]{myers-scotton-duelling-1993}).

The language production process according to \citet[116--119]{myers-scotton-duelling-1993} involves four steps. As the first step, in order to meet the requirements of the communicative situation, speakers take, mainly unconsciously, intentional and socio-pragmatic decisions at the conceptual level. Steps two and three concern the functional level. Step two involves building the frame into which content morphemes are inserted. Specifically, matrix-language lemmas are selected from the speaker's mental lexicon in accordance with her conceptual specifications. At step three, the selected lemmas send information to the ``formulator'', or processing centre, which regulates grammatical encoding procedures. These operations are considered responsible for controlling matrix-language specifications for system morphemes. Myers-Scotton remarks that concrete morphemes may yet be actualised at a later stage, ``nearer the surface'' (p. 118). She further asserts that the essential procedures carried out in the formulator can be covered by the matrix language hypothesis and both the Morpheme Order and the System Morpheme principles. Once the frame is built, lexemes attached to lemmas are realised, and a unified structure is produced. This final step in the production process concerns the positional level. This implies that information about the surface structure is activated. The only structures that violate this model are \textbf{embedded-language islands}, defined as well-formed embedded-language constituents occurring in a matrix language clause. That is, embedded-language content morphemes appear in this case together with embedded-language system morphemes in a clause framed by the matrix language. \citet[119]{myers-scotton-duelling-1993} claims that embedded-language islands are produced ``when ML procedures are entirely inhibited by EL procedures''. The model  distinguishes between obligatory and optional embedded-language islands. Obligatory embedded-language islands are triggered by incongruent morphosyntax. (The idea of obligatory embedded-language islands is elaborated further as the EL Island Trigger hypothesis, see below.) Moreover, not only the matrix language, but also the embedded language can be subject to inhibition. The inhibition of the embedded language is crucial to the System Morpheme Principle as well as the Blocking Hypothesis.

Whilst in the case of the System Morpheme Principle, a filter in the formulator prohibits embedded-language system morphemes, in the case of the Blocking Hypothesis, inhibition applies to embedded-language content morphemes. The \textbf{Blocking Hypothesis} postulates that ``[i]n ML + EL constituents, a blocking filter blocks any EL content morpheme which is not congruent with the ML with respect to three levels of abstraction regarding subcategorization'' \citep[120]{myers-scotton-duelling-1993}. The following kinds of incongruence are considered relevant: first, a mismatch in the morpheme status, i.e., the matrix language uses a system morpheme to code a given grammatical category, whereas the embedded language employs a content morpheme for the same purpose; second, incongruence between embedded-language and matrix language content morphemes in respect of thematic role assignment; third, a mismatch between embedded-language and matrix language content morphemes regarding their discourse or pragmatic functions. Although the Blocking Hypothesis predicts the blocking of any incongruent content morpheme, the presented analysis covers only the cases of incongruent pronouns and prepositions. Pronouns, for example, may be realised as content morphemes in one language and as system morphemes, i.e., as clitics and dummy pronominals, in the other. \citet[126--128]{myers-scotton-duelling-1993} argues that such lack of congruency in the morpheme status is the reason why pronouns analysed as content morphemes do not occur in mixed constituents. However, the discussion of the Blocking Hypothesis is silent on what content morphemes with divergent patterns of thematic role assignment or different discourse or pragmatic functions could be blocked.

The final hypothesis underlying the matrix language frame model is the \textbf{EL Island Trigger Hypothesis}. It predicts when obligatory embedded-language islands must occur:  ``Activating any EL lemma or accessing by error any EL morpheme not licensed under the ML or Blocking Hypotheses triggers the processor to inhibit all ML accessing procedures and complete the current constituent as an EL island'' \citep[139]{myers-scotton-duelling-1993}. The examples provided as evidence of the EL Island Trigger Hypothesis include insertions of English noun phrases modified by demonstrative or possessive pronouns, i.e., system morphemes. The following example illustrates the case of  possessive pronouns: 

\ea{\label{ex:1:10}}
Swahili-English \citep[141]{myers-scotton-duelling-1993}\\
\gll Tu-na-m-let-e-a \textit{our brother} wa Thika\\
	\textsc{1pl-prog}-him-take-\textsc{appl-indic} {}  of Thika\\
\glt `We are taking [it] to our brother of Thika.'
\z

\noindent Myers-Scotton regards the modifier \textit{our} in (\ref{ex:1:10}) as a trigger for the corresponding island because the order of this modifier and its head in English, the embedded language, is at odds with the order of the corresponding constituents in Swahili, the matrix language. According to the production model assumed as well as the System Morpheme Principle, the lemmas which correspond to system morphemes come from the matrix language as early as step two of the production process, i.e., in the formulator, where the frame is built. Consequently, the only possible explanation for the activation of the EL lemma supporting the system morpheme \textit{our} is by error. Crucially, it is owing to this morpheme that the whole EL island is produced. The analysis of a similar case in \citet{myers-scotton-matching-1995} assumes an alternative scenario: at first, the lemma corresponding to the head of the nominal phrase is activated in the mental lexicon and then ``[t]his lemma activates morphosyntactic procedures in the formulator, such that ML procedures are inhibited for the maximal category projection (here, NP) associated with that lemma. The result is an EL island''  (p. 995). In this case, the Embedded-Language Island Trigger hypothesis is dismissed. As such, the explanation by triggering would be more straightforward provided that linearity is possible at an abstract level. I assume that an approach to language production taking into consideration the linear character of speech could arrive at a more realistic account of instances like (\ref{ex:1:10}) than a purely top-down production model.

A specific type of embedded-language islands, according to \citet[142, 144]{myers-scotton-duelling-1993} include set phrases. Unfortunately, she does not specify the diagnostic features of a set phrase.

The Embedded-Language Island Trigger hypothesis appears to be problematic inasmuch as it considers the access to any EL morpheme as erroneous. The regular occurrence of EL islands in such language pairs as English and Spanish \citep[cf.][]{poplack-sometimes-1980}, or Dutch and French \citep[cf.][]{treffers-daller-mixing-1994}, could hardly be only due to erroneous access, the incongruence of the embedded-language material with the matrix language specifications \citep[250]{milroy-lexically-1995}, or the formulaic character of the morpheme string involved. After all, in the model version outlined in \citet{milroy-lexically-1995}, the Embedded-Language Island Trigger hypothesis is reformulated with no mention of erroneous access (p. 249), and  \citet{myers-scotton-matching-1995} discuss embedded-language islands without a reference to the aforementioned hypothesis. \citet[137]{myers-scotton-duelling-1993} acknowledges that embedded-language islands are ``the potential Achilles' heel of the MLF model''; therefore, her later work (\citeyear{myers-scotton-matrix-2001}; \citeyear{myers-scotton-contact-2002}), including joint research with Jake (\citeyear{myers-scotton-matching-1995}), focuses on mechanisms constraining embedded-language islands. The ideas that were initially presented in their \citeyear{myers-scotton-matching-1995} paper ``Matching lemmas'' laid the ground for the Abstract Level model.

\subsection{The Abstract Level model}
As indicated above, the work by \citet{myers-scotton-matching-1995} examines structures that result from either the Blocking Hypothesis or the Embedded-Language Trigger Hypothesis of the matrix language frame model, i.e., the so-called compromise strategies: embedded-language islands, bare forms and \textit{do}-constructions. First and foremost, the Abstract Level model is aimed at explaining these phenomena. Second, it is claimed to shed light on the structure of entries in the mental lexicon.

\citet{myers-scotton-matching-1995} proceed from the premise that abstract grammatical structure contained in lemmas underlying lexical items is distributed at three levels: (i) the level of lexical-conceptual structure (semantic/pragmatic features), (ii) the level of predicate-argument structure, and (iii) the level of morphological realisation patterns. According to the production model assumed \citep[23--25, 76--78]{myers-scotton-contact-2002}, these levels are activated in the following way. At first, the speaker's intentions select a language-specific semantic-pragmatic feature bundle at the conceptual level, which in its turn elects a lemma underlying a content morpheme. If a lemma is activated that supports an embedded-language content morpheme, \citet{myers-scotton-matching-1995} hypothesise that this lemma is matched for congruence against a matrix language counterpart lemma at every level of abstract grammatical structure mentioned above. As the matrix language may lack a counterpart for the given embedded-language lemma, congruence checking is assumed possible against Lexical Knowledge (``generalized but specific to the matrix language'', \citealt[97]{myers-scotton-contact-2002}). The idea that lemmas are checked for congruence in bilingual production is another premise of the Abstract Level model. In case of sufficient congruence between the embedded-language lemma and the matrix language counterpart at all three levels, the embedded-language content morpheme will surface as fully integrated into the matrix language frame, that is, a mixed constituent will be produced. If there is a lack of congruence at one of the levels of abstract grammatical structure, a bare form or an embedded-language island will emerge. It is necessary to note that \citet[][]{myers-scotton-matching-1995} no longer maintain the division of embedded-language islands into obligatory and optional, which was present in \citet{myers-scotton-duelling-1993}. In essence, the Abstract Level model ascribes the use of compromise strategies to the lack of congruence in one of the levels of abstract grammatical structure. Below, I will discuss mismatches between lemmas at all the three levels of abstract grammatical structure.

Following the model, a compromise strategy may be employed when there are differences in lemmas' lexical-conceptual structure, or their semantic and pragmatic features. We can thus assume, for instance, that the embedded-language island in (\ref{ex:1:11}) is produced because there is not sufficient congruence between the lemmas in the lexical-conceptual structure.
\ea{\label{ex:1:11}}
Kazakh-Russian \citep[41--42]{muhamedowa-untersuchung-2006}\\
\gll 
Mustafa ata-m on bes žas-ï-nda \textit{protiv} \textit{basmačestv-a} ket-ken.\\
	Mustafa grandpa-\textsc{poss.1sg} ten five year-\textsc{poss3sg-loc} against basmachi.movement-\textsc{gen.sg.n} go-\textsc{prf3sg}\\
\glt `My grandpa Mustafa fought against the basmachi movement at fifteen.' 
\z

\noindent According to \citet{muhamedowa-untersuchung-2006}, Kazakh has no counterpart for the Russian preposition \textit{protiv} `against', and the speaker's use of this preposition triggers the production of the embedded-language island. Following the scenario envisaged by \citet[995]{myers-scotton-matching-1995}, the choice to use Russian should be made at the conceptual level, although with consequences for the morphological realisation pattern. This is because Kazakh and Russian take different patterns to encode spatiotemporal relations: Kazakh postpositions contrast with Russian prepositions. The observed incongruence at the conceptual level is argued to account for the occurrence of the embedded-language island.

Similarly, the lack of congruence at the functional level, i.e., in either predicate-argument structure, or morphological realisation patterns, is regarded as a motivation for the appearance of embedded-language islands and bare forms. The example below illustrates the occurrence of a bare form, which is hypothesised to be the result of a mismatch in the predicate-argument structure.

\ea{\label{ex:1:12}}
Croatian-English \citep[213]{hlavac-second-generation-2003}\\
\gll {\dots} to je možda najbolj-e što sam ja, što ja \textit{remember}-∅ \textit{yeah}.\\
	{} that be.\textsc{prs.3sg} perhaps best-\textsc{nom.sg.n} \textsc{rel.acc} be.\textsc{prs.1sg} \textsc{1sg.nom} \textsc{rel.acc} \textsc{1sg.nom} \phantom{remember}-∅ \textsc{ptcl}\\
\glt `\dots that is perhaps the best thing that I have, that I remember, yeah.'
\z

\noindent Here, the English verb \textit{remember} and its Croatian counterpart \textit{sjećati se} exhibit divergent argument structures: the Croatian reflexive verb realises the theme as a genitive NP, whereas the English \textit{remember} assigns the theme the non-nominative case. The theme in (\ref{ex:1:12}) is an accusative NP, expressed by the relative pronoun \textit{što} `that'; yet, it is the genitive form of this pronoun \textit{čega} that concurs with the required configuration. The use of the bare verb \textit{remember} seems to be a compromise strategy that resolves the apparent problem of incongruence.\footnote{
\citet[213]{hlavac-second-generation-2003} analyses the pronoun \textit{što} as a nominative form. As the latter exhibits form syncretism, it can be interpreted as expressing either the nominative or the accusative case. Here, it seems more appropriate to assume the accusative case because the nominative NP is also realised in the clause by the personal pronoun \textit{ja}.}
An alternative account of instances like (\ref{ex:1:12}) could be one that allows for linearity at an abstract level. In this case, we could assume that the bare form is triggered by the preceding structure, incongruent with the argument configuration of the Croatian verb but with a higher degree of congruence with the configuration of the English verb. In other words, the speaker, without finding \textit{le mot juste} in Croatian, has to make a compromise and switch to English.

With regard to the third level of abstract grammatical structure, the authors claim that insufficient congruence in patterns of morphological realisation employed by the languages should bring about the occurrence of an embedded-language island. This seems to be the case in the following example:

\ea{\label{ex:1:13}}
Finnish-English \citep[226]{lehtinen-analysis-1966}\\
\gll \dots ja sitte eh \textit{in the afternoon} isä vai minä men-i{\dots}\\
	\phantom{\dots}and then \textsc{hes} {} father[\textsc{nom.sg}] or \textsc{1sg.nom} go-\textsc{pst.3sg}\\
\glt `\dots and then in the afternoon father or I went{\dots}'
\z

\noindent We can hypothesise that the embedded-language island \textit{in the afternoon} in (\ref{ex:1:13}) arises because the morphological patterns that express spatiotemporal location in Finnish and English diverge: this meaning is expressed by prepositions in English, but by postpositions in Finnish. The apparent conflict in the patterns is resolved in that the embedded language is produced.

Although the proposed model seems well elaborate to explain the emergence of mixed constituents as well as many instances of bare forms and embedded-languages islands in code-mixing with various languages, some of its aspects may be considered problematic. First, the model seems capable of predicting the use of compromise strategies, but it remains inexplicit about when a particular strategy is followed, i.e., when a mismatch in congruence will result in an embedded-language island, and not in a bare form. Second, as the use of compromise strategies is explained \textit{ex-post}, it is unclear which level of abstract linguistic structure matters when. For instance, we can assume for examples (\ref{ex:1:11}) and (\ref{ex:1:13}) that the mismatch relevant for the occurrence of embedded-language islands may be at the level of morphological realisation patterns as well as at the level of lexical-conceptual structure. Finally, embedded-language islands and bare forms may occur even when the features of lemmas supporting the respective morphemes match, or when some features of the morphosyntactic structure are shared by both languages. For example, \citet{treffers-daller-mixing-1994} provides instances of embedded-language islands structured as PPs, such as (\ref{ex:1:14}), which can be explained by neither incongruence in lexical-conceptual and predicate-argument structure, nor by incongruent morphological realisation patterns.

\ea \label{ex:1:14}
Dutch-French \citep[208]{treffers-daller-mixing-1994}\\
\gll Wat gaa-t ge do-en \textit{chez} \textit{ce-s} \textit{pauvre-s} \textit{vieux}?\\
	what go-\textsc{prs.2sg} \textsc{2sg} do-\textsc{inf} at \textsc{dem-pl} poor-\textsc{pl} old[\textsc{pl}]\\
\glt `What are you going to do in the house of these poor old people?'
\z

\noindent In (\ref{ex:1:14}), the French PP \textit{chez ces pauvres vieux} `at these poor old people's (place)' is an embedded-language island. Apparently, the island and its Dutch equivalent \textit{bij die arme oudjes} do not demonstrate incongruence at any level of abstract grammatical structure: The concept `old people' exists in both languages and is encoded in a similar way, i.e., by nouns formed by conversion from adjectives (i.e., Dutch \textit{oud} and French \textit{vieux}). With regard to morphological realisation patterns, both French and Dutch use prepositions and pre-nominal determiners, and in the present case they both rely on overt plural markers (the demonstratives are plural forms: \textit{ces} and \textit{die}, and Dutch also marks plural on the noun \textit{oudje} with the suffix \textit{-s} and on the adjective \textit{arm} with the suffix \textit{-e}). Therefore, it is necessary to examine other factors that may influence the emergence of embedded-language islands. As I will show in the subsequent chapters, the frequency with which words co-occur in the embedded language can explain occurrences of embedded-language islands more effectively.

\subsection{The 4-Morpheme model}
A more recent development in the work by Myers-Scotton is the 4-Morpheme model (\citeyear{myers-scotton-matrix-2001}; \citeyear[73--82]{myers-scotton-contact-2002}; for a recent overview, see \citealt[][]{myers-scotton-jake16}). As was the Abstract Level model, this is a submodel of the matrix language model. The 4-M model is motivated by the need to provide a more precise account of ``certain types of congruence problems [that] arise in codeswitching between certain language pairs'' \citep[42]{myers-scotton-matrix-2001} and in specific contact phenomena, such as creole formation, language attrition and second language acquisition (which abound with counter-examples to the original matrix language frame model, proposed in \citealt{myers-scotton-duelling-1993}). The 4-M model introduces a subdivision of system morphemes based on their distribution in code-mixing. The classification relies on the premise that system morphemes are activated at two different levels. Myers-Scotton claims that the proposed morpheme classification is valid for language production in general.

By and large, the model distinguishes between four types of morphemes: content morphemes, early system morphemes, bridge late system morphemes and outside late system morphemes. The classification builds on three abstract oppositions. The first opposition concerns the conceptual activation of lemmas underlying morphemes. The hypothesis here is that some of the lemmas have a more direct connection to speaker's intentions than others. Under this opposition, represented as [±conceptually activated], content morphemes and early system morphemes have the feature [+conceptually activated], whereas the other two morpheme types, i.e., late system morphemes, have the feature [--conceptually activated]. \citet{myers-scotton-contact-2002} claims that speakers' intentions activate language-specific semantic/pragmatic feature bundles, which select lemmas in the mental lexicon supporting content morphemes. Such a [+conceptually activated] element has semantic content. Lemmas that underlie content morphemes are hypothesised to be elected directly. Directly elected lemmas may in their turn activate other lemmas at the same level. These other lemmas are thus elected indirectly and underlie early system morphemes. In other words, morphemes  which are activated at the level of the mental lexicon (and have the feature [+conceptually activated]) are subdivided into content morphemes and early system morphemes depending on whether they are elected directly or not. This opposition is formalised as [±thematic role]. \textbf{Content morphemes} have the feature [+thematic role], and \textbf{early system morphemes} are [--thematic role]. The prototypical content morphemes are nouns and most verbs. Early system morphemes express the \textit{phi}-features of person, number and gender and therefore include determiners and plural morphemes.

According to \citet{myers-scotton-contact-2002}, ``Together, the lemmas underlying content morphemes and early system morphemes send directions to the Formulator to build larger linguistic units'' (p. 77). This is how lemmas underlying late system morphemes are activated. Depending on the type of grammatical information carried by the activated lemma supporting a late system morpheme, Myers-Scotton distinguishes between bridge late system morphemes and outside late system morphemes. \textbf{Bridge late system morphemes} integrate content morphemes and early system morphemes into larger constituents, whereas \textbf{outside late system morphemes} provide for coindexical relations across maximal projections. Formally, this opposition is represented like this: [±refers to grammatical information outside of Maximal Projection of Head]. While outside late system morphemes refer to grammatical information outside of Maximal Projection of Head, bridge late system morphemes do not refer to grammatical information outside of the maximal projection of their heads. \citet[75]{myers-scotton-contact-2002} illustrates `bridges' with English possessives \textit{of} and \textit{'s} and the French \textit{de}, as in \textit{beaucoup de gens} `a lot of people'. ``Outsiders'' are exemplified by case affixes and morphemes expressing subject-verb agreement.

In a nutshell, the model differentiates between the following four types of morphemes on the basis of the proposed abstract oppositions:
\begin{itemize} 
\item content morphemes: [+conceptually activated], [+thematic role];
\item early system morphemes: [+conceptually activated], [--thematic role];
\item bridge late system morphemes: [--conceptually activated], [--grammati-cal information outside of Maximal Projection of Head]
\item outside late system morphemes: [--conceptually activated], [+grammatical information outside of Maximal Projection of Head]
\end{itemize}

Myers-Scotton introduces an important amendment concerning the order of morpheme activation in the case of fusion of grammatical features. In some languages, grammatical features may be fused under the same morpheme. For example, German determiners express gender, number and case simultaneously. Russian and other Slavic languages abound with such portmanteau morphemes: inflections within the Russian nominal declensional system encode number, gender and case, and also the category of animateness. \citet[][]{myers-scotton-contact-2002} calls such portmanteau morphemes multimorphemic elements (p. 305), or multimorphemic lexemes (p. 81). The hypothesis regarding multimorphemic elements is this (p. 305): %\citep[305]{myers-scotton-contact-2002}

\begin{quote}In multimorphemic elements (consisting of two or more system morphemes and including a late system morpheme), the late system morpheme takes precedence. This means that the entire element shows distribution patterns as if it were a late system morpheme. This is the `pull down' or `drag down' principle.
\end{quote}
In other words, a late system morpheme such as one expressing case pulls down the portmanteau morpheme which also expresses gender and number. As such, Russian nominal inflections should be classified as outside late system morphemes because morphemes expressing case are the last to be activated of the discussed morpheme types.

However, structures examined in \citet{muhamedowa-untersuchung-2006} seem to provide evidence against this hypothesis. The following is an example of Kazakh Russian discussed by the author:
\ea \label{ex:1:15}
Kazakh Russian \citep[92]{muhamedowa-untersuchung-2006}\\
\gll edinstvenn-yj ja by-l-a semejn-yj i eščë s det'-mi\\
	sole-\textsc{nom.sg.m} \textsc{1sg.nom} be-\textsc{pst-sg.f} with.family-\textsc{nom.sg.m} and additionally with children-\textsc{inst.pl}\\
\glt `I was the only one with a family and what is more with children.'
\z

\noindent Before presenting the argument, I will mention the relevant features of the Russian verbal phrase. First, in the past tense subject-verb agreement involves the inflectional values of number and gender. Second, predicative adjectives also agree with the subject in inflectional values and are additionally assigned either the nominative, or the instrumental case. Two standard-Russian versions of (\ref{ex:1:15}) are \textit{Edinstvennaja ja byla semejnaja i eščë s det'mi} or \textit{Edinstvennoj ja byla semejnoj i eščë s det'mi}.  In both versions, we can observe agreement in number and gender between the verb \textit{byla} and the predicative adjectives \textit{edinstvennaja}, or \textit{edinstvennoj}, and \textit{semejnaja}, or \textit{semejnoj}, respectively. Following the hypothesis given above, the nominal inflections here are late outside system morphemes because they refer to grammatical information outside of the maximal projection of their heads. Yet, the verbal phrase in (\ref{ex:1:15}) lacks the necessary agreement: the inflections of the predicative adjectives correspond to the nominative singular masculine, and not the nominative (or instrumental) singular feminine. We can thus state that agreement between the verb and the predicate adjectives is only in number and case, and is thus partial. Therefore, (\ref{ex:1:15}) could be considered a violation to the aforementioned hypothesis and further to the suggested classification according to abstract oppositions.\footnote{
According to \citet[92]{muhamedowa-untersuchung-2006}, the speaker in (\ref{ex:1:15}) is Russian-dominant, she acquired Kazakh late, as a second language. The speaker produces well-formed predicative adjectives in the same conversation. However, numerous instances of gender neutral Russian adjectives are common in Kazak-Russian code-mixing, especially as modifiers in Russian noun phrases inserted in Kazakh \citep[cf.][77--93]{muhamedowa-untersuchung-2006}.}

Despite the identified problems with the matrix language frame model, it is necessary to emphasise the importance of the seminal work by Myers-Scotton and associates, which opened up the field towards psycholinguistic theorising by linking code-mixing phenomena to existing language production models.

The analysis of insertional code-mixing reported in the remainder of this book will build on the matrix language-embedded-language dichotomy, without adopting the specific assumptions of the Matrix Language Frame model as well as the underlying production model. I will thus distinguish between mixed constituents, or minimal insertions, and embedded-language islands, or maximal, multimorphemic insertions. My principal focus will be on the nature of embedded-language islands, whose pervasive use in bilingual speech has been explained by their their special status in the bilingual mental lexicon/grammar. Multimorphemic insertions have been argued to correspond to units in the  mental lexicon/grammar, which are comparable with entries of mono-morphemic words. According to this account, multimorphemic sequences with unit status are perceived and produced online as mono-morphemic words. To this approach I turn next.

\section{Multimorphemic units in insertional code-mixing}{\label{unit-hypothesis}}
An account of embedded-language islands as multimorphemic units in the lexicon is proposed by Ad Backus (\citeyear{backus-two-1996}, \citeyear{backus-evidence-1999}, \citeyear{backus-units-2003}). His analysis of code-mixing proceeds from the matrix language-embedded-language asymmetry, crucial for the MLF model, but it differs from that model substantially. While acknowledging the semantic basis for distinguishing between content and system morphemes, the MLF model, he criticises, excludes semantic factors; he contends that ``they are not presented as such'' \citep[115]{backus-two-1996} in the MLF model. By integrating semantic and structural factors, \citet{backus-evidence-1999, backus-units-2003} manages to shed light on the nature of embedded-language islands. The approach presented in his \citeyear{backus-two-1996} monograph and subsequent publications (\citeyear{backus-evidence-1999}, \citeyear{backus-units-2003}) draws on premises from Langacker's Cognitive Grammar (\citeyear{langacker-foundations-1987,langacker-foundations-1991}) and Goldberg's Construction Grammar (\citeyear{goldberg-1995}). In this section, I will first outline the premises relevant for the current overview and then introduce the features of multimorphemic units. Finally, I will discuss the unit hypothesis and the ``conceptual unit'' hypothesis put forth by Backus.

One premise postulates that the grammar of a language and its lexicon are not separate modules but form a continuum. The lexicon is considered an inventory of symbolic units, i.e., form-meaning pairings. Some of these units are abstract schemas, which are broadly equivalent to morphosyntactic rules in other frameworks, while others are specific (lexical) items. Schemas, or constructions, sanction novel combinations of lexical items. These two types of units form the poles of the lexicon-grammar continuum.

Another premise concerns the nature of lexical units: ``lexical units can be of any length and complexity'' \citep[84]{backus-units-2003}. Therefore, whether morpheme-sized or multimorphemic, all units are supposed to be accessed directly in production and not composed on-line, even if multimorphemic units can be decomposed into their constituent parts \citep[cf.][129]{backus-two-1996}. Support for this view comes from psycholinguistic experimental research, notably from behavioural studies on processing of multimorphemic units: noun plurals and idiomatic expressions. Whilst \citet{baayen-dijkstra-schreuder} provide evidence that Dutch speakers store high-frequency regular noun plurals (see \ref{UBL}, for further details), \citet{libben-multidetermined-2008}, and \citet{tabossi-processing-2008} underpin the role of decomposition in idiom comprehension. The study by \citet{libben-multidetermined-2008} confirms that decomposition plays only a marginal role in idiom comprehension, and \citet{tabossi-processing-2008} show that semantic compositionality of idiomatic expressions does not influence their syntactic flexibility. 

From these premises it follows that a string of morphemes or words, sanctioned by a schema, gains the status of a lexical item once it is entrenched, i.e., when it receives an allocated mental representation. This scenario asks for a definition of multimorphemic lexical units. \citet{backus-units-2003} defines \textbf{units} as ``any recurrent combinations of two or more morphemes that together exhibit idiomatic meaning'' (p. 90).\footnote{
The term ``idiomatic meaning'' is not restricted to idioms alone, it should be understood broadly, as an equivalent to `non-compositional meaning' \citep[cf.][86]{backus-units-2003}.} Further lexical units include high-frequency composite forms as well as forms with irregular morphosyntax. That is, in order to qualify as units, multimorphemic elements have to either (1) demonstrate irregular morphosyntax (e.g., the past \textit{caught} and the plural \textit{children}), (2) express non-compositional meaning (e.g., the collocation \textit{play tennis}, the idiom \textit{this is a piece of cake}, meaning `X can be accomplished with ease', and the discourse marker \textit{the thing is}), or (3) be of high frequency (e.g., the regular past forms \textit{worked} and \textit{helped}). Moreover, multimorphemic units can be discontinuous and have open slots. For instance, the aforementioned idiom \textit{this is a piece of cake} allows for variability in the first two elements: the first element is an open slot filled by a nominal phrase, and the second element is any form of the copula \textit{be}. This idiom can hence be represented formally like this: [X BE \textit{a piece of cake}]. The open slots accompanying discourse markers, such as \textit{the thing is}, are filled by clauses. As open slots can possibly be instantiated by an unlimited range of lexical items, only their frozen elements are used as a diagnostic feature of units.

In his work on Dutch insertions in Turkish sentences, \citet{backus-evidence-1999} claims that maximal insertions consisting of co-occurring embedded-language, i.e., Dutch, morphemes, frequently correspond to Dutch multimorphemic lexical units. \citet{backus-units-2003} articulates this idea as the \textbf{unit hypothesis}, which stipulates that ``[e]very multimorphemic EL [= embedded-language] insertion is a unit, inserted into a ML clausal frame'' (p. 91). This hypothesis is plausible; bilingual corpora abound in instances of lexical-unit insertion, for instance:

\ea \label{ex:1:16}
Hindi-English \citep[228]{bhatt-1997}\\
\gll kəl hi \textit{Israel} \textit{government} ne kəha ki Asaad\textsubscript{i} \textit{peace talks} ke prət̪i \textit{serious} nəhɪ̃ː hai aur pro\textsubscript{i} \textit{political} \textit{games} khel rəha hai\\
yesterday \textsc{ptcl} {} {} \textsc{erg} said that {} {} {} toward {} not is and {} {} {} play \textsc{prog} is\\
\glt `Only yesterday the Israel government said that Asaad is not serious about peace talks and that he is (instead) playing political games.'
\z

\noindent Here, the word strings \textit{Israel government}, \textit{peace talks} and \textit{political games} are lexical units inserted into Hindi clauses, the first two present compound nouns and the third is a collocation. The compound \textit{Israel government} is one of the few instantiations of the pattern [COUNTRY \textit{government}]. This pattern is productive only to some degree: in \citetitle{coca} \textit{(COCA)}\footnote{
Unfortunately, I could not use \citetitle{ice-ind} because of its modest size; the word sequence \textit{Israel government} is not attested in it.
}, only 16 lexemes appear as realisations of the first pattern element\footnote{These nouns are as follows: \textit{Hong Kong}, \textit{Singapore}, \textit{Pakistan}, \textit{Taiwan}, \textit{New Zealand}, \textit{Iraq}, \textit{Kuwait}, \textit{Sudan}, \textit{Afghanistan}, \textit{Bangladesh}, \textit{Israel}, \textit{Botswana}, \textit{Myanmar}, \textit{Palau}, \textit{Flanders}, \textit{Cameroon}.
}, with the exception of initialisms such as \textit{US} and \textit{UK} and plural forms such as \textit{United States} and \textit{Seychelles}. The pattern is restricted to nouns denoting country names whose corresponding adjectives end in \textit{-i} or \textit{-ese}. Whereas the schemas [\MakeUppercase{country-adj} \textit{government}] and [\MakeUppercase{country}\textit{'s government}] -- illustrated by word strings \textit{Israeli government} and \textit{Israel's government} -- are the common, productive ways to refer to `the government of a country', the pattern used in (\ref{ex:1:16}) can be applied only to a limited set of lexemes, resulting in a small, presumably unproductive set of word combinations \citep[cf.][74]{bauer01}. It is thus possible to analyse the word sequence \textit{Israel government} as a specific lexical item. The word string \textit{peace talks} is the other inserted compound noun in (\ref{ex:1:16}), it is a \textit{plurale tantum} containing the suffix -\textit{s} and thus a frozen morphological form. This feature enables one to consider this compound noun as a lexical unit. Finally, the collocation \textit{political games} is distinguished by its idiomatic meaning and also qualifies as a lexical unit.

The three instances of multiword unit insertion in (\ref{ex:1:16}) coincide with syntactic phrases. However, multimorphemic units in the embedded language do not have to fit the constituent structure neatly. Such units can encompass more than one syntactic phrase. \citet{backus-units-2003} demonstrates this point by providing numerous examples of the predicate-complement construction, which is lexically realized as recurrent collocations of the embedded language. This type of insertional code-mixing may be found in other bilingual corpora, for instance:
\ea \label{ex:1:17}
Moroccan Arabic-Dutch \citep[245]{boumans-syntax-1998}\\
\gll n-dir-u \textit{pauze} \textit{houd-en}?\\
1-do-\textsc{pl} break take-\textsc{inf}\\
\glt `Shall we take a break?'
\z

\ea \label{ex:1:18}
Moroccan Arabic-French \citep[315]{bentahila-davies-1983}\\
\gll tajbqa j-\textit{confronter} \textit{ces} \textit{idées}\\
keep.\textsc{3sg} \textsc{imprf}-oppose these ideas\\
\glt `He keeps opposing these ideas.'
\z

\ea \label{ex:1:19}
Tamil-English \citep[80]{sankoff-et-al-1990}\\
\gll anta  \textit{car}-ei \textit{drive} paNNanum\\
that { \hspace{4mm}-\textsc{acc}} {} do.must\\
\glt `We must drive that car.'
\z

\noindent The multimorphemic insertions in these examples can be analysed as lexical units. In (\ref{ex:1:17}), the Dutch word string \textit{pauze houden} `take a break' is inserted into the Arabic clausal frame and forms an embedded-language island. Syntactically, the string is an instantiation of the predicate-complement construction; from the lexico-grammatical perspective, it is an idiomatic collocation. \citet[246]{boumans-syntax-1998} contends that the collocational status of the string explains the absence of a determiner before the noun, which is obligatory in Dutch usage. The sentence in (\ref{ex:1:18}) allows for two different analyses. Following the MLF model, one has to admit that \textit{j-confronter} `oppose' is a mixed constituent and the nominal phrase \textit{ces id\'{e}es} `these ideas' is an embedded-language island. Yet, following the unit hypothesis an analysis is preferred according to which the whole sequence in French is a lexical unit, and only one of its elements acquires an Arabic morpheme. Support for the latter treatment of the instance comes from the corpus-driven dictionary \citetitle{wortschatz}, which lists the nouns \textit{idées} `ideas', \textit{réalité} `reality' and \textit{expériences} `experience' as the most frequent collocates of \textit{confronter} `oppose'. (\ref{ex:1:19}) is a similar case: the lexical items \textit{car} and \textit{drive} are inserted into the Tamil matrix structure. While the noun receives the Tamil accusative suffix, the verb does not take any infinitive suffix required in spoken Tamil (i.e., neither \mbox{-\textit{kka}} nor -\textit{a}, \citealt[cf.][73]{schiffman-tamil-1999}), and is thus analysed as a bare form. From the lexico-grammatical perspective, however, the two words form a collocation: the noun \textit{car} is the most frequent collocate of the verb \textit{drive}.\footnote{
This fact is confirmed by the observations of nominal collocates of the verb \textit{drive} in the \citetitle{bnc} corpus and \citetitle{coca} \textit{(COCA)}.}
The insertion of \textit{car} and \textit{drive} can thus be regarded as unit insertion, even though the unit is separated by the Tamil suffix \textit{-ei}. 

The presented evidence, even if sporadic, is in favour of the unit hypothesis. Nevertheless, owing to the broadness of the generalisation that the hypothesis expresses, the acceptance of the hypothesis would be unreasonable if any embedded-language morpheme sequence inserted in the context of the matrix language is granted unit status \citep[cf.][42]{wray-2002}. However, \citet[91]{backus-units-2003} is aware of the generality of the hypothesis and points to two types of counterexamples \citep[also see][105--107]{backus-evidence-1999}. One type involves multiword embedded-language insertions that are not lexical units in the embedded language. The other type includes insertions of single words instead of expected multimorphemic units. To account for the second type of counterexamples, \citet{backus-units-2003} puts forward the \textbf{``conceptual unit'' hypothesis}. The hypothesis predicts that ``[t]he use of EL conceptual structure in CS [=code-mixing] can, but does not have to, lead to EL units. The actual morphemes do not have to be from the EL. ML morphemes will have semantically basic meanings in such cases'' (p. 92). To illustrate the ``conceptual unit'' hypothesis, I draw on an example from Irish-English code-mixing data:
\ea
\label{ex:1:20}
Irish-English \citep[184]{stenson-1990}\\
\gll Tagann sé isteach \textit{handy}.\\
come-{\textsc{prs}} it inward {}\\
\glt `It comes in handy.'
\z

\noindent We can regard (\ref{ex:1:20}) as a partial loan translation of the corresponding English idiomatic expression \citep[184]{stenson-1990}. In line with the ``conceptual unit'' hypothesis, the matrix language items -- here Irish -- express semantically basic meanings, whereas the item with a specific meaning is realised in English. However, \citeauthor[]{backus-units-2003} provides counterexamples for his second hypothesis as well. These include cases in which a collocation element bearing a specific meaning is in the matrix language, for example:

\ea{}\label{ex:1:21}
Turkish-Dutch \citep[111]{backus-units-2003}\\
\gll {} \textit{koffie} dök\\
Dutch: \textit{koffie} \textit{inschenken}\\
\glt {\textcolor[rgb]{1,1,1} k} \hspace{7.5mm} `pour coffee'
\z

\noindent \citet[113]{backus-units-2003} asserts that possible reasons for the occurrence of such mixed collocations are the grammatical incongruence between the languages, as in (\ref{ex:1:21}), or processing factors such as the effect of recency. At the same time he admits that not every counterexample can be explained.

Corroborative evidence regarding the role of multimorphemic units in code-mixing is offered by \citet{treffers-daller-2005}, who analyses Dutch-French code-mixing involving compounds and nominal groups, and by \citet[67]{muhamedowa-untersuchung-2006}, who discusses the insertion of plural forms in Kazakh-Russian code-mixing. Moreover, \citeauthor{myers-scotton-2006}, in her more recent work (\citeyear{myers-scotton-2006}), acknowledges that two types of lemma entries exist in the lexicon, she states that ``[e]ntries for some content morphemes combine with other entries cross-linguistically in `fast and clean' on-line production. The entries for some other content morphemes are parts of holistic multimorphemic units that are readily integrated into ML frames\dots'' (p. 211). Despite the support for Backus's hypotheses, one type of counterexamples for the unit hypothesis are not handled explicitly in his approach. As mentioned above, such examples include multiword embedded-language insertions that are not lexical units in the embedded language. In other words, determining unit boundaries, except when relying solely on code-mixing data, is problematic. Specifically, it can be difficult to decide whether an embedded-language word string in the matrix language context corresponds to one lexical unit, as the hypothesis predicts, or a combination of such units. For instance, several units can be posited for the multimorphemic insertion in (\ref{ex:1:22}): 

\ea{}\label{ex:1:22}
Ewe-English \citep[22]{amuzu-2013}\\
\gll Míe \textit{download} \textit{journal} \textit{article} ma-wo katã xle nyitsɔ\\
1\textsc{pl} {} {} {} that-\textsc{pl} all read a.day.removed\\
\glt `We downloaded all those journal articles and read [them] a few days ago.'
\z

\noindent One possibility, following the unit hypothesis, is to regard the string \textit{download journal article} as a unit. Another option is to analyse the string as consisting of two units \textit{download} and \textit{journal article}. A third possibility is to presume that the string is a blend of two collocations combined online \textit{journal article} and \textit{download article}. Remember that in order to qualify as a unit, the string has to either convey a non-compositional meaning or be very frequent. These criteria make the first analysis implausible, i.e., the string does not have a non-compositional semantics, nor do its parts co-occur frequently. Whether the second, or the third analysis should be adopted needs further exploration.

The problem with this specific instance and both hypotheses is the need for verification testing \citep[cf.][42]{wray-2002}. Although I relied on corpus data as additional evidence when I discussed some of the multimorphemic insertions above, the evidence provided is clearly unsystematic. A systematic examination of the hypotheses will involve an analysis of a bilingual corpus. Nevertheless, my random analysis of instances of mixing as well as the evidence provided by other researchers point at the plausibility of the hypotheses articulated by \citet{backus-units-2003}. To put the unit hypothesis to a test and to gather evidence in its favor or against it through a systematic study of a bilingual corpus is the aim of this book. I will explore multimorphemic insertions by comparing them with mono-morphemic, or minimal, insertions in a bilingual corpus and by analysing their distributions in large monolingual corpora.

An issue that may have ramifications for the analysis of minimal insertions in a bilingual corpus is their status in the lexicon. Some studies of bilingual speech classify single-word insertions as pertaining to either code-switching/mixing, or borrowing \citep[e.g.,][]{poplack18}. Other analyses abandon this distinction, either partially \citep[e.g.,][78-81]{muysken-bilingual-2000}, or altogether \citep[e.g][]{backus-13,backus-cs-15}. In the next section, I will present and discuss the various views on this issue.

\section{Insertional code-mixing versus lexical borrowing}\label{CM-borrowing}

There is a widespread agreement in the field that in a synchronic analysis of code-mixing, minimal insertions fall under either switched, or borrowed words  \citep[e.g.,][]{haugen-1953-vol1,poplack-etal-1988, myers-scotton-duelling-1993,milroy-code-switching-1995,thomason-language-2001}. Borrowed items differ from switched items in their status: the former are not alien but nativised in the variety of the language spoken in a bilingual community. The process of word nativisation, i.e., its introduction and assimilation in the recipient language, can only be studied diachronically \citep[cf.][]{poplack-dion-2012} and is thus beyond the scope of this study. However, a synchronic differentiation between switched and borrowed forms, as I will argue, is possible and useful. (\citealt{backus-cs-15} is a proponent of a dynamic approach to code-mixing and borrowing, which integrates the synchronic and  diachronic aspects of these phenomena.) A confusion of switched forms with native forms may lead to a skewed database \citep[164]{myers-scotton-duelling-1993}, resulting eventually in an inadequate analysis. As argued by \citet{poplack-comment-2011}, the division of lone embedded-language items into borrowed and switched should not be done \textit{a priori} and has thus to rely on the these items' specific characteristics. Before I discuss the diagnostic features of borrowed forms, I will briefly sketch the typology of language mixing phenomena proposed by Poplack and associates. 

The approach developed by Poplack and presented in her volume \citeyear[][]{poplack18} \citetitle[][]{poplack18} assumes three types of phenomena: code-switching (equivalent to my use of the term "mixing"), nonce borrowing and established borrowing. The first criterion for distinguishing between code-switching and borrowing is the number of words switched: a multiword sequence is an ``unambiguous'' code-switch. A lone other-language item may either be a switch or a borrowing \citep[cf.][2]{poplack-comment-2011}. If this item is integrated morphosyntactically into the recipient language, it qualifies as a borrowing \citep[cf.][]{sankoff-et-al-1990}. Morphosyntactic integration is the second criterion used in this classification. A bilingual speaker may borrow an item for a moment, or the whole bilingual community can use it on a regular basis. While the borrowed form in the former case is classified as  a nonce borrowing, it is regarded as an established borrowing, or a loanword, in the latter case. That is, the final criterion, on which the typology is based, involves the processes of spread and nativisation. The category of borrowing as used in at the beginning of this section will thus correspond to Poplack and collaborators' category of \textbf{established borrowing}.

In order to embed their classification in a more general discussion of borrowing, it is useful to contrast their typology with the central categories adopted in the Matrix Language Frame model. Poplack's multiword switches are akin to Myers-Scotton's embedded-language islands. Nonce-borrowings correspond to mixed, or bilingual, constituents, and morphosyntactically unintegrated single-word switches are identical to bare forms. Finally, the category of established borrowing coincides with Myers-Scotton's category of lexical borrowing \citep[cf.][163--170]{myers-scotton-duelling-1993}. These correspondences are given in Table (\ref{tab:1:2}).

\begin{table}
\begin{small}
\begin{tabular}{m{5em}m{5em}m{5em}lm{5em}lm{6em}} 
\midrule
\addlinespace[2mm]
{Author} &multiword insertion &  \multicolumn{5}{l}{Single-word insertion} \\
\addlinespace[2mm]
\cmidrule{3-3}
\cmidrule{5-7}
\addlinespace[2mm]
 & & {unintegrated} & & \multicolumn{3}{l}{integrated} \\
\addlinespace[2mm]
\cmidrule{5-5} 
\cmidrule{7-7}
\addlinespace[2mm]
  & & & & sporadic & & recurrent \mbox{(and nativised)} \\
\addlinespace[2mm]
\midrule
\addlinespace[2mm]
 Myers-Scotton &embedded-language island &bare form & &mixed constituent & & lexical \mbox{borrowing}\\
\addlinespace[2mm]
 Poplack &code-switch &code-switch & &nonce borrowing & & established loanword\\
\addlinespace[2mm]
\midrule
\end{tabular}
\end{small}
\caption{Contrasting comparison of the approaches to language mixing by \citet[][]{poplack18} and \citet[][]{myers-scotton-contact-2002}.}
\label{tab:1:2}
\end{table}

Despite the differences between these approaches in the methodological procedures applied and the theoretical issues raised, the structures that both approaches identify and investigate exhibit a high degree of overlap. However, similarities end as soon as we examine how lexical borrowings are identified on synchronic grounds. A lack of agreement on this issue is characteristic of the whole field.

Researchers who distinguish between switched and (established) borrowed forms apply the following diagnostics: morphological integration into the base language \citep{macswan-2000}, recurrence \citep{poplack-sankoff-1984, myers-scotton-duelling-1993}, diffusion in the community \citep{poplack-etal-1988,poplack18}, listedness \citep{milroy-code-switching-1995, muysken-bilingual-2000, muhamedowa-untersuchung-2006, stammers-deuchar-2012} and speakers' acceptance \citep{poplack-sankoff-1984}. Whilst \citet{macswan-2000} treats all Spanish words integrated morphologically into Nahuatl as borrowings, \textbf{morphological integration} is not a sufficient criterion for \citet{myers-scotton-duelling-1993}. She claims that incomplete morphological integration may apply to both borrowed and switched forms \citep[191]{myers-scotton-duelling-1993}. According to \citet[183--188]{myers-scotton-duelling-1993}, languages can use several patterns for marking a particular feature: one pattern presupposes full morphological marking, whereas another involves partial marking, or a lack of marking. Consequently, when applied to the morphological integration of foreign words, these patterns result in either full or incomplete integration. Myers-Scotton refers to these two types of marking as central and peripheral. Elaborating on the idea of central and peripheral marking, \citet[52--53]{boumans-syntax-1998} links different marking types to productivity; he states that some of the morphological processes in a language are more, or less, productive than others. That is, a borrowing can demonstrate a lack of integration when it follows a specific, morphologically unproductive pattern \citep[cf.][]{dressler04}. \citet[45]{muhamedowa-untersuchung-2006} illustrates this point by considering a case of morphological non-integration in Russian: foreign nouns with a vowel in the stem-final position, such as \textit{kak\'{a}o} `cocoa', \textit{k\'{o}fe} `coffee', \textit{pal't\'{o}} `coat' and the like, form a class of indeclinable nouns within the Russian nominal system. Thus, these long-established loans cannot be regarded as fully integrated into the Russian morphological system. To take another example, German verbs borrowed from Latin or French which include long-established loans ending in \textit{-ieren}, such as \textit{regieren} `rule', \textit{probieren} `taste', \textit{studieren} `study' and many others, are integrated into the German verbal system also only to a degree. Unlike the majority of German verbs, they do not take the prefix \textit{ge-} as part of the circumfix in the form of past participle, following the rule that \textit{ge-} attaches only to initially stressed items (cf. the pair \textit{mach-en} `make-\textsc{inf}' -- \textit{ge-mach-t} `\textsc{ptcp}-make-\textsc{ptcp}' and  the pair \textit{regier-en} `rule-\textsc{inf}' -- \textit{regier-t} `rule-\textsc{ptcp}'). These examples show that established loanwords may follow a certain pattern in a language which does not allow for full morphological integration. Note that the material used for illustrations includes forms pertaining to monolingual grammars. Therefore, only a synchronic analysis of a monolingual system, possibly supplemented with a diachronic analysis, can enable inferences about the degree of integration. I argue that when bilinguals produce switched or borrowed forms on-line, they can draw on all patterns available to them, regardless of their status, whether central or peripheral. In other words, if a peripheral, or unproductive recipient language pattern is used to accommodate an item from another language, this item should be analysed as well-integrated, for the produced mixed form concurs with the morphosyntactic requirements of the recipient language \citep[cf.][]{sankoff-et-al-1990}.\footnote{
Note that the difference in the approaches is also manifested in the grammatical domains where integration is investigated: \citet{myers-scotton-duelling-1993} and \citet{boumans-syntax-1998} refer to morphological integration alone, whereas Poplack and associates \citep[e.g.,][]{sankoff-et-al-1990, meechan-orphan-1995} proceed from integration in the morphosyntax of a language. Furthermore, whilst the former regard integration as a matter of degree, the latter view it as an abrupt process.}
In this respect, a synchronic analysis of code-mixing alone does not seem appropriate for determining productive patterns of a language, and the only possible outcome of such an analysis is the identification of forms that either exhibit morphosyntactic integration or lack it \citep[cf.][]{poplack-dion-2012}. As such, morphosyntactic integration is not an adequate criterion to distinguish between insertional mixing and (established) loanwords, but a prerequisite for a form to qualify as a borrowing \citep[cf.][]{poplack18}.

As \citet{myers-scotton-duelling-1993} refutes morphological integration as a criterion for distinguishing borrowing from insertional mixing, she makes the division between the two categories by virtue of \textbf{recurrence}.
She asserts that ``CS [=code-switching/mixing] forms have little recurrence value, in contrast with B [=borrowed] forms'' \citep[163]{myers-scotton-duelling-1993} and suggests frequency of occurrence in absolute and relative values as a reliable criterion. Her book \textit{Duelling languages} contains two case studies which operationalise recurrence as frequency, these studies investigate the realisation of numerals in Shona-English code-mixing and the use of English \textit{because} and \textit{but} as borrowed forms in Shona. The discussion of the results entails concern regarding relative frequency: like every researcher dealing with the factor `frequency', Myers-Scotton is eager to know ```how much' relative frequency is {`enough'}{''} (\citeyear[204]{myers-scotton-duelling-1993}), and admits that setting the threshold is an arbitrary decision. Despite the expressed reservation, she views the distinction between borrowed and switched forms  as crucial for the matrix language frame model because borrowed forms, and not switched forms, are projected by lemmas tagged for the matrix language \citep[41]{myers-scotton-contact-2002}. 

A discussion of recurrence as a determinant of (established) borrowings would be incomplete without mention of the pioneering work by \citet{poplack-etal-1988}. The authors perform a large-scale analysis of lone other-language items in a vast corpus of French-English bilingual speech, of a size unmatched by any other bilingual corpus to date. The usage frequency of a borrowing is related to its social integration and defined by the number of speakers using it. Established loans -- corresponding to ``borrowings" in the terminology adopted in this section -- are thus distinguished by \textbf{diffusion} in the speech community. Specifically, \citet[55]{poplack-etal-1988} handle borrowed items as widespread loans if they are uttered by more than ten speakers whose speech is represented in the corpus. The authors acknowledge that the threshold for this criterion, i.e., ten people, is the result of an ``arbitrary, though rather severe'' decision \citep[100]{poplack-etal-1988}. To avoid a terminological confusion, it is necessary to emphasise that \citeauthor[]{poplack-etal-1988}'s use of the term ``frequency of use" deviates from Myers-Scotton's considerably: for Myers-Scotton, frequency of use is the frequency with which a borrowed item occurs in a corpus of bilingual speech, regardless of the number of people who utter it. Both conceptions of frequency as a criterion for distinguishing borrowing from mixing have been criticised in a number of studies \citep[e.g.,][]{haust-codeswitching-1995, boumans-syntax-1998,muhamedowa-untersuchung-2006, stammers-deuchar-2012}.

Whilst one problem with word frequency that \citet[49]{haust-codeswitching-1995}, \citet[57]{boumans-syntax-1998} and \citet[46]{muhamedowa-untersuchung-2006} identify is that coincidental circumstances such as the conversation topic and the speech style disparage frequency distribution in smaller corpora -- a common source of material in the field -- a more serious objection is raised by \citet[57]{boumans-syntax-1998} and \citet{backus-13}, who assert that a borrowed word in a specific community can be a switched item for a subset of speakers of this community. Beyond that, \citet[][29]{backus-13} takes an extreme position and states that the mixing-borrowing ``debate is misguided, because a foreign-origin word can be both: borrowing and codeswitching [=code-mixing] are not mutually exclusive like that''. These considerations make \citet[96--97]{backus-two-1996} and \citet[58--60]{boumans-syntax-1998} abandon the distinction between these phenomena as hopeless. Nonetheless, other researchers, for example,  \citet[][]{muhamedowa-untersuchung-2006}, maintain this distinction and rely on listedness as the factor determining the status of loanwords.

\textbf{Listedness} as a distinctive feature of borrowing is introduced in \citet{milroy-code-switching-1995}. \citet{milroy-code-switching-1995} defines borrowing as a process ``which involves the incorporation of lexical elements from one language in the lexicon of another language'' (p. 189). Hence, he views code-mixing as a supra-lexical phenomenon and refers to borrowing, whether momentary or established, as a sublexical phenomenon. Muysken's approach resembles that of Poplack and associates. However, in \citet{milroy-code-switching-1995, muysken-bilingual-2000}, the differentiation between established and nonce borrowings is based on the criterion ``listedness'', i.e., being part of a memorised list. This means that an (established) loanword has to be listed in the lexicon of a speaker after the corresponding speech community has accepted it. According to \citet{milroy-code-switching-1995, muysken-bilingual-2000}, conventionalisation may also affect code-mixing. Namely, specific multiword sequences in the source language can appear in the recipient language discourse on a regular basis. multiword loans are common in monolingual use: for instance, numerous Latin expressions such as \textit{ex post}, \textit{ad hoc}, \textit{tabula rasa}, \textit{anno domini}, \textit{et caetera} have been adopted by many European languages. \citet{lantto2015} examines conventionalised word sequences in a situation of bilingualism. Using data from the Spanish Basque Country she shows that Spanish multiword discourse markers are conventionalised to such an extent that they infiltrate most bilingual conversations. When multiword strings, represented in the mental lexicon as units, acquire a high degree of diffusion in the community, they can behave like established borrowings. Thus, care should be taken in analyses of code-mixing regarding multiword strings because these strings are also possible candidates for conventionalisation.

\citet{muhamedowa-untersuchung-2006} and \citet{stammers-deuchar-2012} use the criterion ``listedness'' to differentiate between code-mixing and borrowing. An operationalisation of this criterion includes a reinterpretation of listedness as a property of the mental lexicon: the authors reinterpret it as dictionary attestation. This operationalisation proves useful especially in the context of long-standing language contact, such as the situations of bilingualism between Welsh and English \citep{stammers-deuchar-2012} and Kazakh and Russian \citep{muhamedowa-untersuchung-2006}. However, such operationalisation is problematic for languages and varieties spoken in immigrant settings in view of the absence of lexicographic sources. In the aforementioned study, \citet{poplack-etal-1988} also considered dictionary attestation as a possible determinant of social integration of English lexical items in the French-speaking community. In doing so, they systematically sought the examined English lexical items in numerous dictionaries of Canadian and European French. The result of their analysis \citep[58--59]{poplack-etal-1988} is that dictionary attestation is not a reliable predictor of loanword status: 18\% of frequent (i.e., used by more than ten speakers) English words in French discourse from their corpus were not attested in the corresponding dictionaries. At the same time, some words listed in those dictionaries had the status of momentary borrowings in their data. In contrast to this study, \citet{stammers-deuchar-2012} test listedness, operationalised as dictionary attestation, and token frequency as predictors of full morphological integration of English verbs in Welsh and come to the opposite conclusion. In Welsh, the initial consonants of verbs employ soft mutation depending on the lexico-grammatical context. According to the authors, borrowed verbs achieve full morphosyntactic integration only if they use consonant mutation and thus completely assimilate into the Welsh system. Their analysis shows that only verbs attested in the corresponding dictionary rely on soft mutation to a degree comparable with native Welsh verbs. High-frequency verbs that are not attested in the corresponding dictionary do not employ this morphophonological alternation. Thus only ``listed'' verbs are considered \textit{bona fide} established loans. My interpretation of this result is twofold. For one, \citeauthor{stammers-deuchar-2012}'s result can be possibly explained by the duration of language contact. In a cross-linguistic study of foreign lexeme integration, \citeauthor{nortier-schatz} (\citeyear{nortier-schatz}, quoted in \citealt[][53]{boumans-syntax-1998}) argue that the integration of foreign lexemes can be related to the duration of language contact. They find that in a situation of long-established contact, as that between Spanish and Quechua in South America, lexeme integration is higher than in a situation of recent contact, such as that between Dutch and Moroccan Arabic in The Netherlands. We can assume that English verbs attested in the Welsh dictionary are subject to the examined morphophonological alternation because they entered Welsh earlier than current high-frequency borrowings, which are still not affected by this process. In other words, the lexical items may correspond to different historical layers of the vocabulary and can hardly be comparable. Another explanation of the restricted applicability of Welsh soft mutation to English verbs frequently occurring in Welsh lies in the nature of stem alternation. \citet[][216]{haspelmath-sims} state that ``the effects of morphophonological alternations need not be found in loanwords'', even if a particular stem alternation is productive (e.g., Turkish \textit{k} / \textit{ğ} and Indonesian Nasal Substitution, ibid.: 219). The examined alternation, like Nasal Substitution in Indonesian, might be no longer productive when applied to novel words as such, which is why unattested frequent English items remain unaffected. Crucially, determining the productivity of the alternation examined by \citet{stammers-deuchar-2012} will require obtaining psycholinguistic evidence. 

The overview of various approaches to borrowing confirms that there are good reasons to differentiate between mixed and borrowed forms in a synchronic analysis of code-mixing. One reason is that borrowed items, whether at the sublexical or supra-lexical level, apparently have the same status in the mental lexicon as native items. Hence, ignoring this distinction would result first in a skewed database and eventually in a misleading analysis of code-mixing. The overview of the studies tackling the problem of identifying borrowing demonstrates that there is no single criterion applicable to all contact situations and all databases. Dictionary attestation is impossible for languages spoken in immigrant communities, for example in the Russian-speaking community in Germany. Control over diffusion of foreign lexemes in the community, a correlate of their social integration, sets the most demanding requirement for compiling vast corpora of bilingual speech.\footnote{
This is one reason why the findings by \citet{poplack-etal-1988} have not been validated in other corpora to date.}

In this situation, the only feasible criterion is recurrence, operationalised as token frequency in absolute and relative values. The research reported in the subsequent chapters will thus consider usage frequency of items at both the sublexical and the supra-lexical levels. That is, I will examine single words, but also multiword sequences in the bilingual corpus with regard to their usage frequencies since all of them may be possible candidates for social integration in the speech community.

\section{Conclusion}

In this chapter, I have presented current approaches to the linguistic patterning in bilingual speech. According to the widely accepted typology articulated by Muysken, the major types of language mixing include insertion, alternation, and congruent lexicalisation. Following Muysken, I have argued that there is a correlation between a specific type of mixing in a given community and a particular constellation of linguistic, psychological and social factors, although the social conditions of language use take precedence over other factors. For example, I showed that in immigrant settings the dominant type of mixing depends on the bilingual individual's generational membership in an immigrant community, which is in its turn related to the individual's bilingual proficiency and, in the end, to their linguistic experience as such. The speech of intermediate-generation immigrants, who are usually fluent in both the community and the country language, exhibits variable patterns of insertional mixing, with minimal and maximal insertions alternating with each other.

Minimal insertion, described as a process whereby content morphemes from another language are combined with  recipient language markers, is a pervasive process of word incorporation in language contact. There is general agreement among scholars that this is a default mode of insertional mixing. At the same time, views vary widely on the nature of maximal insertion, or the occurrence of embedded-language islands, and the factors contributing to their emergence in bilingual speech. While the proponents of the Matrix Language Frame model argue that maximal insertions arise because there is a lack of overlap between the lexico-grammatical aspects of the equivalent structures in two languages, the cognitive-linguistics approach to bilingual speech proposed by Backus assumes maximal insertions to correspond to lexical units in the mental lexicon/grammar. The unit status makes maximal insertions highly accessible to bilingual individuals in online speech production and hence contributes to their ubiquity in bilingual speech. I have argued that although this suggestion seems perfectly plausible, it requires further research and systematic evaluation.

The chapter has also discussed insertion and lexical borrowing as distinct phenomena of bilingual speech. Despite adverse views aired on this dichotomy, most researchers agree that these are different, though related phenomena and should thus be distinguished in analyses of bilingual speech. However, controversy exists over the criteria indicative of lexical borrowing. In light of this debate, I have presented arguments in favour of the view that recurrence, or frequency of use, may be operationalised as a reliable diagnostic feature of lexical borrowing.

%(\citealt[e.g.,][]{lyons-language,fasold-introduction,behrens-language,varvaro-2010,auer-sprachwissenschaft,finegan15})
